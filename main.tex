\documentclass[12pt, a4paper]{report}

% *1. Configuración de Idioma y Fuentes (Guía Docente)

 \usepackage{fontspec}
 \usepackage{microtype}
 \usepackage{polyglossia}
 \usepackage[version=4]{mhchem}
 \setmainlanguage{spanish}
 \setotherlanguage{english}

  % ?  REQUISITO FCQ: Fuente Times New Roman (Debe estar instalada en el sistema)
   \setmainfont{TIMES.TTF}[
    Path = ./fonts/,
    BoldFont = TIMESBD.TTF,
    ItalicFont = TIMESI.TTF,
    BoldItalicFont = TIMESBI.TTF]


%* 2. Formato de Página (Guía Docente)

  \usepackage[left=2.5cm, right=2.5cm, top=2.5cm, bottom=2.5cm]{geometry} % MÁRGENES 2.5 CM (REQUISITO FCQ)
  \usepackage{setspace}
  \onehalfspacing % INTERLINEADO 1.5 LÍNEAS (REQUISITO FCQ)

  %? PÁRRAFOS: Sin sangría, con espacio vertical entre ellos
   \setlength{\parindent}{0pt} % ELIMINA LA SANGRÍA (REQUISITO DEL USUARIO)
   \setlength{\parskip}{1em}  % Espacio vertical entre párrafos

  %? Para incluir PDFs (Portada y Declaración IA)
    \usepackage{pdfpages} 
    \usepackage{afterpage}
    \usepackage[hidelinks]{hyperref}

%* 3. Configuración de Capítulos (titlesec)
  \usepackage{caption} 
  \captionsetup[table]{name=Tabla}  
  \usepackage{titlesec}

  %? REQUISITO USUARIO: Capítulo sin la palabra "Capítulo", sin espacios grandes
    \titleformat{\chapter}[hang]{\normalfont\bfseries\Huge}{\thechapter.}{0.5em}{\bfseries\Huge} 
    \titlespacing*{\chapter} {0pt}{0pt}{1.5\baselineskip} % Sin espacio antes (0pt), sin espacio después (0pt), pero dejamos un espacio de una línea y media antes del texto para legibilidad.

%* 4. Contenido Académico y Citas

  \usepackage{amsmath, amssymb}
  \usepackage{siunitx} % Unidades
  \DeclareSIUnit{\angstrom}{\text{\AA}} % Definición del símbolo Angstrom
  \usepackage{graphicx}
  \usepackage{wrapfig}
  \usepackage{subcaption} 
  \usepackage{longtable}

%* 4. Configuración de Citas y Bibliografía (APA Numerado)

  \usepackage[
    backend=biber,
    style=apa,        % Usamos el estilo APA para la bibliografía
    sorting=nyt,      % Ordenar por Nombre, Año, Título
    language=spanish,
    citestyle=numeric % Forzamos que las citas en el texto sean números [1]
  ]{biblatex}

    %* 4.1. Quitar la "sangría francesa" típica de APA (margen izquierdo a 0)
 \setlength{\bibhang}{0pt}

    %* 4.2. Añadir espacio entre cada referencia
    \setlength{\bibitemsep}{1.5\baselineskip}
    \ExecuteBibliographyOptions{labelnumber}
    \DeclareFieldFormat{labelnumberwidth}{\mkbibbrackets{#1}}
    \defbibenvironment{bibliography}
    {\list
        {\printtext[labelnumberwidth]{%
        \printfield{labelnumber}}}
        {\setlength{\labelwidth}{\labelnumberwidth}%
      \setlength{\leftmargin}{\labelwidth}%
      \setlength{\labelsep}{\biblabelsep}%
      \addtolength{\leftmargin}{\labelsep}%
      \setlength{\itemsep}{1.5\baselineskip} 
      \setlength{\parsep}{\bibparsep}}}
    {\endlist}
    {\item}

 \addbibresource{Bibliografia.bib}

%* 5. PERSONALIZACIÓN DE FIGURAS (REQUISITO TUTORA: Figura 1, 2, 3...)

  \usepackage{chngcntr}
  \counterwithout{figure}{chapter}
  \counterwithout{table}{chapter}
  \usepackage{enumitem}
  \usepackage{multirow}
  \usepackage{booktabs}
  \usepackage{tabularx}
  \usepackage{float}

  % ? Solución para que los enlaces NO se salgan del margen ---
    \setcounter{biburllcpenalty}{7000}
    \setcounter{biburlucpenalty}{8000}
    \setcounter{biburlnumpenalty}{9000}

%? META-DATOS DEL DOCUMENTO
  \hypersetup{
    pdftitle={ESTUDIO DE ALGUNAS ESTRUCTURAS DE COMPUESTOS INORGÁNICOS Y MODIFICACIONES DE SIMETRÍA}, % Pon el título exacto de tu TFG
    pdfauthor={Lucia Lan Burillo Blazquez},       % Tu nombre completo
    pdfsubject={Trabajo de Fin de Grado - Grado en Química}, % La asignatura o el tipo de trabajo
    pdfkeywords={Cristalografía, Lantánidos, VESTA, Zircón, Monacita, Xenótimo, Rhabdofano, Simetría}, % Palabras clave importantes
    pdfcreator={LaTeX},                 % Opcional: indica con qué se creó
    pdfproducer={luLaTeX},             % Opcional
    pdfdisplaydoctitle=true             % Muestra el título del documento en la barra de ventana en vez del nombre del archivo
  }

%* 6. INICIO DEL DOCUMENTO

\begin{document}


% !A. PORTADA (Num. no visible)

%! REQUISITO FCQ: Insertar la portada como PDF (DOCX convertido a PDF)
%!\includepdf[pages=1]{gq-portada-tfg-2024-25.pdf} 

\clearpage

%! DECLARACIÓN RESPONSABLE DE AUTORÍA (OBLIGATORIO)
%! REQUISITO FCQ: Insertar el documento oficial (firmado) como PDF
%!\includepdf[pages=1, pagecommand={\thispagestyle{plain}}]{declaracion.pdf} 


% ÍNDICES (COMIENZAN LA NUMERACIÓN )
\pagenumbering{arabic}
\setcounter{page}{1}
\tableofcontents
\clearpage

% C. CUERPO PRINCIPAL DEL TFG

\chapter{Introducción}
\setmainlanguage{english}
\captionsetup[table]{name=Table}   % Fuerza que se llame "Table" en esta sección
\captionsetup[figure]{name=Figure} % Fuerza que se llame "Figure" en esta sección

\section{Theoretical Background}
A crystal is defined as a solid material whose atoms are arranged in an orderly repeating pattern extending in all three spatial dimensions \cite{ashcroft1976solid}. The study of crystal structures is fundamental in materials science, as the arrangement of atoms directly influences the physical and chemical properties of materials \cite{callister2020materials}.

This periodic order is defined by the unit cell in different directions around each atom. The directions are non-coplanar and originate at each atom of the crystal. The axes are chosen by symmetry criteria. Once the axes are defined, the unit cell is established as the smallest portion of the crystal that, when repeated in space through translations along the axes, can recreate the entire crystal structure. Its geometry is defined by reticular parameters: the lengths of the edges (\(a\), \(b\), \(c\)) and the angles between them (\(\alpha\), \(\beta\), \(\gamma\)).

There are seven crystal systems based on the relationships between the unit cell parameters: cubic, tetragonal, orthorhombic, hexagonal, trigonal, monoclinic, and triclinic. Each system is characterized by specific constraints on the lengths and angles of the unit cell.

\begin{table}[H]
  \centering
  \caption{Summary of the seven crystal systems with their characteristic unit cell dimensions and interaxial angles.}
  \label{tab:crystal_systems}
  \vspace{0.1cm}
  \begin{tabular}{l c c}
    \toprule
    System       & Dimensions          & Angles                                                  \\
    \midrule
    Cubic        & \(a = b = c\)       & \(\alpha = \beta = \gamma = 90^\circ\)                  \\
    Tetragonal   & \(a = b \neq c\)    & \(\alpha = \beta = \gamma = 90^\circ\)                  \\
    Orthorhombic & \(a \neq b \neq c\) & \(\alpha = \beta = \gamma = 90^\circ\)                  \\
    Hexagonal    & \(a = b \neq c\)    & \(\alpha = \beta = 90^\circ\), \(\gamma = 120^\circ\)   \\
    Rhombohedral & \(a = b = c\)       & \(\alpha = \beta = \gamma \neq 90^\circ\)               \\
    Monoclinic   & \(a \neq b \neq c\) & \(\alpha = \gamma = 90^\circ\), \(\beta \neq 90^\circ\) \\
    Triclinic    & \(a \neq b \neq c\) & \(\alpha \neq \beta \neq \gamma \neq 90^\circ\)         \\
    \bottomrule
  \end{tabular}
\end{table}

The arrangement of atoms within the unit cell defines 14 distinct Bravais lattices, categorized into seven crystal systems. These lattices are distinguished by their centering: primitive (P), body-centered (I), face-centered (F), and base-centered (C).

In primitive (P) lattices, atoms are located exclusively at the corners of the unit cell. Body-centered (I) lattices feature atoms at the corners plus an additional atom at the geometric center of the cell. Face-centered (F) lattices have atoms positioned at the corners and at the center of each face, whereas base-centered (C) lattices contain atoms at the corners and at the centers of two opposite faces.

The 14 Bravais lattices are illustrated in Figure \ref{fig:Bravais_lattices}.

\begin{figure}[H]
  \centering
  \includegraphics[width=0.7\textwidth]{Imagenes/Bravais_Lattices.jpeg}
  \caption{The 14 Bravais lattices classified by crystal system and centering type (P: primitive, I: body-centered, F: face-centered, C: base-centered).}
  \label{fig:Bravais_lattices}
\end{figure}

Lattice planes are defined by Miller indices (hkl), which are a set of three integers that denote the orientation of the planes in the crystal lattice. These indices are derived from the reciprocals of the fractional intercepts that the plane makes with the crystallographic axes.

Symmetry operations are transformations that map a crystal structure onto itself, preserving its overall arrangement. These operations include rotations, reflections, inversions, and translations. The combination of these symmetry operations defines the space group of a crystal, which provides a comprehensive description of its symmetry properties.

\begin{itemize}
  \item Rotations:
        \begin{itemize}
          \item Rotational axes \((n)\) are the ones around which the crystal can be rotated by specific angles and still appear unchanged. Common rotation axes include 2-fold (180°), 3-fold (120°), 4-fold (90°), and 6-fold (60°) rotations.
          \item Rotation-reflection combined symmetry operations involve a rotation followed by a reflection across a plane. These operations can create complex symmetry elements in crystals.
        \end{itemize}
  \item Reflections:
        \begin{itemize}
          \item Mirror planes \((m)\) are planes that divide the crystal into two symmetrical halves, where one half is the mirror image of the other.
        \end{itemize}
  \item Inversions:
        \begin{itemize}
          \item Inversion centers (\(\bar{1}\) or \(i\)). Inversion through a central point maps every point \((x, y, z)\) to \((-x, -y, -z)\).
          \item Rotoinversion axes \((\bar{n})\) involve a rotation followed by an inversion through a point. Common rotoinversion axes include \(\bar{3}\) (rotation + inversion) and \(\bar{4}\).
        \end{itemize}
  \item Translations: represent a shift of the entire crystal lattice by a specific vector.
        \begin{itemize}
          \item Helicoidal axes (Screw axes) represent a combined rotation and translation along the axes. \(X_n\) where \(n < X\). Types of screw axes include:
                \begin{itemize}
                  \item 2\(_1\): 180° rotation + translation of 1/2 along the axes.
                  \item 3\(_1\): 120° rotation + translation of 1/3 along the axes.
                  \item 3\(_2\): 240° rotation + translation of 2/3 along the axes.
                \end{itemize}
          \item Glide planes combine reflection with translation parallel to the plane. Common types include \(a\)-, \(b\)-,and \(c\)- glides (translation \(1/2\) along the corresponding axis), the \(n-\) glide (\(1/2\) along face diagonal), and \(d\)-glide (diamond glide, (\(1/4\)) translation along the face diagonal).
        \end{itemize}
\end{itemize}

Platonic solids are highly symmetrical, three-dimensional shapes with identical faces, edges, and angles. In crystallography, the five Platonic solids (tetrahedron, cube, octahedron, dodecahedron, and icosahedron) can be used to describe the coordination environments of atoms within a crystal structure.

Point groups classify the symmetry of objects based on their rotational and reflectional symmetries, excluding translations. There are 32 distinct point groups in three-dimensional space, each characterized by a unique combination of symmetry elements. Hermann-Mauguin notation is a symbolic system used to describe the symmetry elements of crystals, including point groups and space groups. This notation provides a concise way to represent the symmetry operations present in a crystal structure.

The 230 space groups classify the symmetry of crystal structures, combining translational and point symmetries. Each space group is identified by a unique number and symbol, providing a comprehensive description of the symmetry operations that can be applied to the crystal lattice.

International crystallographic tables indicate for each atom the following information:
\begin{itemize} [label=\textendash]
  \item Coordinates indicate the position within the unit cell, expressed as fractions of the cell parameters (\(a\), \(b\), \(c\)).
  \item Occupancy factors represent the proportion of a specific atomic site that is occupied, ranging from 0 (completely unoccupied) to 1 (fully occupied).
  \item Isotropic thermal parameters (\(U\)) describe the average displacement of   atoms from their mean positions due to thermal vibrations, assuming uniform movement in all directions.
  \item Wyckoff positions describe the specific locations of atoms within a unit cell based on the symmetry of the space group. Each Wyckoff position is associated with a multiplicity (the number of equivalent positions generated by symmetry operations) and a letter (such as \(a\), \(b\), \(c\), etc.) that indicate the symmetry of the site, \(a\) corresponding to the highest symmetry site in the unit cell. For example, \(4a\) indicates four equivalent positions with the highest symmetry.
  \item Symmetry indicates whether there is or not (\(1\)) a symmetry element with its symbol.
\end{itemize}

\section{State of Art}
The study of the crystal structures presented above is particularly relevant when analyzing lanthanide orthophosphates (\ce{LnPO_4}), a class of materials with rich structural chemistry and diverse technological applications. These compounds are fundamental in fields such as photonics, catalysis, radioactive waste management, and biomedical imaging \cite{rafiuddin_review_2022}. The selection of the specific crystalline structures—whether the anhydrous form (Monazite, Xenotime) or the hydrated form (Rhabdophane, Churchite)—is critical, as the crystal structure dictates key properties such as luminescence intensity, chemical stability, and magnetic behavior \cite{garrido_hernandez_photoluminescence_2016, rafiuddin_structural_2022}.

The inclusion of the Zircon (\ce{ZrSiO_4}) structure in this study is fundamental for two crystallographic and comparative reasons:

\begin{enumerate}
  \item Structural Prototype: Zircon serves as the crystallographic aristotype for the Xenotime phase. Both compounds are isostructural, crystallizing in the same tetragonal system with the \(I4_1/amd\) space group \cite{chong_synthesis_2024} . Understanding the arrangement of \ce{ZrO8} and \ce{SiO_4} polyhedra in Zircon is a prerequisite for analyzing the geometry of heavy lanthanide phosphates (\ce{HREEPO_4}), where \ce{Ln^{3+}} and \ce{P^{5+}} occupy the positions of \ce{Zr^{4+}} and \ce{Si^{4+}}, respectively.
  \item Stability Benchmark: Zircon is widely recognized as a benchmark material for evaluating radiation resistance in nuclear wasteforms \cite{rafiuddin_review_2022}. Comparative studies have demonstrated that under irradiation, silicate minerals with the Zircon structure undergo amorphization and lack structural recovery mechanisms \cite{chong_synthesis_2024}. In contrast, the analogous phosphate structures (Xenotime and Monazite) exhibit ``self-healing'' capabilities and superior resistance to amorphization at lower temperatures, justifying the specific interest in phosphates over silicates for the long-term storage of actinides \cite{chong_synthesis_2024}.
\end{enumerate}

Polymorphism and Stability Domains The crystallization of these phosphates into a specific structure depends primarily on the ionic radius of the lanthanide cation (\ce{Ln^{3+}}) and synthesis conditions, such as temperature and pH \cite{enikeeva_structure_2023}. There is a distinct structural dependence based on the contraction of the lanthanide series:

\begin{figure}[H]
  \centering
  \includegraphics[width=0.5\textwidth]{Imagenes/Chong_etal_2024_radius.jpeg}
  \caption{Dependence of lanthanide orthophosphate structure type on the ionic radius of the \ce{Ln^{3+}} cation (adapted from Chong et al., 2024 \cite{chong_synthesis_2024}).}
  \label{fig:radius_dependence}
\end{figure}

\begin{itemize}[label=\textendash]
  \item Light Lanthanides (\ce{LREE}), e.g., La-Gd: Tend to crystallize in the monoclinic Monazite type structure or its hexagonal hydrated form, Rhabdophane (\ce{LnPO_4*nH_2O}) \cite{chong_synthesis_2024}.
  \item Heavy Lanthanides (\ce{HREE}, e.g., Tb-Lu): Mainly crystallize in the tetragonal Xenotime phase or the monoclinic hydrated Churchite phase (\ce{LnP_4\cdot 2H_2O}) \cite{chong_synthesis_2024}.
  \item Transition Zone: Intermediate elements such as Gadolinium (\ce{Gd}), Terbium (\ce{Tb}), or Dysprosium (\ce{Dy}) can adopt multiple structures depending on conditions. A ``critical radius'' exists for the transition between monazite and xenotime structures, typically located between the ionic radii of \ce{Tb} and \ce{Gd} \cite{mogilevsky_miscibility_2007}.
  \item Recent thermodynamic studies have indicated that rhabdophane phases are generally metastable with respect to the corresponding monazite plus water at all temperatures under ambient pressure; however, rhabdophane often precipitates first due to kinetic controls \cite{shelyug_thermodynamics_2018}.
  \item Phase Transformations and dehydration There is a direct transformation relationship between low-temperature hydrated phases and high-temperature stable anhydrous phases. For instance, the Rhabdophane structure is metastable and, upon heating, loses its structural water (zeolitic water in the channels) to irreversibly transform into the Monazite structure. This transformation typically occurs between 500 and 600 °C, although the dehydration process can begin at lower temperatures \cite{kenges_synthesis_2017,shelyug_thermodynamics_2018}. Similarly, the Churchite phase is stable at low temperatures (typically synthesized via precipitation), but upon heating above 300 °C, it dehydrates and transforms directly into the Xenotime structure \cite{rafiuddin_structural_2022}. Controlling these transformations is vital, as the presence of water and the symmetry of the lanthanide ion's environment drastically affect properties such as photoluminescent efficiency; for example, the removal of \ce{-OH} defects during the transition to the tetragonal phase significantly improves emission intensity in \ce{Tb}-doped systems\cite{garrido_hernandez_photoluminescence_2016}.
\end{itemize}

Technological Interest and Applications Each of these structures offers specific advantages for industrial and scientific applications:

Nuclear Waste Management: Monazite and Xenotime are considered excellent matrices for the long-term confinement of high-level nuclear waste (actinides) due to their high radiation resistance, thermal stability, and low solubility \cite{chong_synthesis_2024,rafiuddin_review_2022}. Monazite ceramics have demonstrated the ability to incorporate significant amounts of tetravalent and trivalent actinides, mimicking natural minerals that have retained radionuclides over geological timescales \cite{shelyug_thermodynamics_2018}.

\begin{figure}[H]
  \centering
  \includegraphics[width=0.6\textwidth]{Imagenes/nucelar_waste.jpeg}
  \caption{Schematic illustration of monazite and xenotime as candidate matrices for long-term immobilization of actinides in nuclear waste disposal.\cite{rafiuddin_review_2022}}
  \label{fig:nuclear_waste}
\end{figure}

Structural Ceramics and Coatings: Due to their high melting points (>2000 °C), chemical stability, and compatibility with other oxides, anhydrous phases are investigated as fiber coatings in Ceramic Matrix Composites (CMCs) to improve fracture toughness and prevent oxidation \cite{chong_synthesis_2024}.

Nanotechnology and Catalysis: The Rhabdophane structure is of significant interest due to its morphology. It has been reported that rhabdophane nanoparticles synthesized by hydrothermal methods can form single-crystal rods with internal mesoporosity or cavities, which is attractive for catalytic and adsorption applications \cite{enikeeva_structure_2023}.

Biomedical Imaging: Recent investigations highlight the potential use of Churchite materials as contrast agents in Magnetic Resonance Imaging (MRI). Magnetic susceptibility measurements have shown that these hydrated phases exhibit effective magnetic moments similar to free lanthanide ions, making them suitable candidates for such applications \cite{rafiuddin_review_2022}.

\begin{figure}[H]
  \centering
  \includegraphics[width=0.5\textwidth]{Imagenes/microscope_garrido.jpg}
  \caption{Scanning electron micrograph showing the characteristic high-aspect-ratio nanorod morphology of hydrated rhabdophane (adapted from Garrido Hernández et al.\cite{garrido_hernandez_photoluminescence_2016}). This morphology is ideal for biomedical probes, unlike the granular shape of the tetragonal phase \cite{chong_synthesis_2024}.}
  \label{fig:biomedical_imaging}
\end{figure}

\chapter{Objetivo}
El objetivo principal de este trabajo es el estudio cristalográfico comparativo de la familia de los ortofosfatos de lantánidos (\ce{LnPO4}). Se analizarán las estructuras del Zircón (\ce{ZrSiO4}) como aristotipo y su relación con las fases anhidras Xenótimo y Monacita, así como sus contrapartes hidratadas Churchita y Rhabdofano. El análisis se realizará mediante el software VESTA \cite{prog:vesta}, enfocándose en cómo la variación del radio iónico y la hidratación inducen modificaciones de simetría y cambios en el entorno de coordinación, determinando así sus propiedades y aplicaciones tecnológicas.

\chapter{Resultados y Discusión}
\captionsetup[table]{name=Tabla}
\captionsetup[figure]{name=Figura}

Empleando el software de visualización y análisis cristalográfico VESTA\cite{prog:vesta}, se generaron las representaciones gráficas de las estructuras cristalinas de los compuestos estudiados. A continuación, se detallan las características estructurales de cada uno de los materiales analizados, incluyendo sus parámetros de celda, coordenadas atómicas, factores de ocupación y parámetros térmicos isotrópicos.

\section{Zircón (\texorpdfstring{\ce{ZrSiO4}}{ZrSiO4})}
El zircón (\ce{ZrSiO4}) constituye el aristótipo estructural de la familia de los neosilicatos tipo zircón y representa un referente fundamental en estudios de estabilidad frente a la radiación. Los parámetros cristalográficos empleados en este estudio proceden de ``Materials Data on \ce{ZrSiO4}'' del Materials Project\footnote{\url{https://next-gen.materialsproject.org/materials/mp-4820}} \cite{zircon_data_2020}. La estructura del Zircón se caracteriza por un sistema cristalino tetragonal con grupo espacial \(I4_1/amd\) (No. 141), de alta simetría ideal para cationes pequeños.

La notación Hermann-Mauguin \(I4_1/amd\) se descompone de la siguiente manera: el símbolo \(I\) denota una red de Bravais centrada en el cuerpo; \(4_1\) representa un eje helicoidal cuaternario con traslación de 1/4 a lo largo de \(c\); la barra oblicua indica que el eje perpendicular al plano de simetría contiene un plano especular tipo \(a\) (deslizamiento axial); \(m\) y \(d\) representan planos especulares y de deslizamiento diagonal, respectivamente. El grupo puntual \(4/mmm\): \(4/m\) indica un eje de rotación de \(90^\circ\) con un plano de simetría horizontal perpendicular al eje principal \(c\). La segunda \(m\) indica un plano de simetría vertical al eje principal que pasa por los ejes \(a\) y \(b\). La tercera \(m\) indica un plano de simetría vertical al eje principal que pasa por las diagonales del plano \(ab\). Con \(Z=4\) unidades de fórmula por celda unitaria.

Las coordenadas atómicas, factores de ocupación y parámetros térmicos isotrópicos para la estructura del Zircón se presentan en la Tabla \ref{tab:coord_Zircón}.

\begin{table}[H]
  \centering
  \caption{Parámetros cristalográficos del zircón (\ce{ZrSiO4}): coordenadas fraccionarias, ocupación, desplazamientos térmicos isotrópicos, posiciones de Wyckoff y simetría puntual.}
  \label{tab:coord_Zircón}
  \vspace{0.1cm}
  \begin{tabular}{l c c c c c c c}
    \toprule
    Átomo & \(x\)  & \(y\)  & \(z\)  & Ocup & \(U\) & Pos. Wyckoff & Sim.          \\
    \midrule
    Zr    & 0.0000 & 0.7500 & 0.1250 & 1    & 0.002 & \(4a\)       & \(\bar{4}m2\) \\
    Si    & 0.0000 & 0.7500 & 0.6250 & 1    & 0.004 & \(4b\)       & \(\bar{4}m2\) \\
    O     & 0.0000 & 0.0661 & 0.1953 & 1    & 0.007 & \(16h\)      & \(.m.\)       \\
    \bottomrule
  \end{tabular}
\end{table}
Los parámetros de la celda unitaria son \(a = b \neq c\) con \(a=b=6.62, c = 6.00\) \si{\angstrom} y los ángulos \(\alpha = \beta = \gamma = 90^\circ\). El volumen de la celda unitaria es \(286.533\) \si{\cubic\angstrom}. Cada celda unitaria contiene 4 unidades de fórmula (\(Z=4\)); por lo tanto, la fórmula empírica es \ce{Zr4Si4O16}. La celda unitaria se muestra en las proyecciones a lo largo de los ejes \(a\), \(b\) y \(c\) en la Figura \ref{fig:zircon_uc}.

\begin{figure}[H]
  \centering
  \begin{subfigure}[b]{0.3\linewidth}
    \includegraphics[width=\linewidth]{Imagenes/1_ZIRCON/ATOMS/ZIRCON_a.png}
    \caption{Eje \(a\)}
    \label{fig:zircon_atoms_a}
  \end{subfigure}
  \hfill
  \begin{subfigure}[b]{0.3\linewidth}
    \includegraphics[width=\linewidth]{Imagenes/1_ZIRCON/ATOMS/ZIRCON_b.png}
    \caption{Eje \(b\)}
    \label{fig:zircon_atoms_b}
  \end{subfigure}
  \hfill
  \begin{subfigure}[b]{0.3\linewidth}
    \includegraphics[width=\linewidth]{Imagenes/1_ZIRCON/ATOMS/ZIRCON_c.png}
    \caption{Eje \(c\)}
    \label{fig:zircon_atoms_c}
  \end{subfigure}
  \caption{Proyecciones de la celdilla unidad del zircón a lo largo de los ejes cristalográficos principales, mostrando la disposición de los átomos de Zr (verde), Si (azul) y O (rojo).}
  \label{fig:zircon_uc}
\end{figure}

\vspace{0.5cm}

\begin{minipage}[c]{0.75\textwidth} % Columna de TEXTO (Izquierda)
  \setlength{\parskip}{1em} % Mantiene espacio entre párrafos
  El entorno de coordinación del silicio define un tetraedro \ce{SiO4} aislado, característico de los neosilicatos. Este poliedro exhibe una perfección cristalográfica notable en cuanto a las distancias de enlace, con cuatro enlaces Si--O equivalentes de \SI{1.626}{\angstrom}, reflejando el sitio de Wyckoff \(4b\) con simetría puntual \(\bar{4}m2\). La uniformidad de las distancias se cuantifica mediante un índice de distorsión de la longitud de enlace nulo (BLD = 0.0000).

  No obstante, el análisis angular revela que el poliedro no corresponde a un tetraedro geométricamente ideal. La varianza angular de \SI{97.71}{\degree\squared} y el alargamiento cuadrático de 1.0238 evidencian una compresión apreciable del poliedro a lo largo del eje \(c\) cristalográfico, como se ilustra en la Figura \ref{fig:zircon_c_poly}. El volumen del tetraedro resultante es de \SI{2.13}{\cubic\angstrom}.
\end{minipage}% <--- ¡IMPORTANTE: No borres este símbolo de porcentaje!
\hfill
\begin{minipage}[c]{0.20\textwidth} % Columna de IMAGEN (Derecha)
  \centering
  \begin{figure}[H]
    \includegraphics[width=\linewidth]{Imagenes/1_ZIRCON/ZIRCON_Td.png}
    \captionsetup{justification=centering}
    \caption{Tetraedro \ce{SiO4} aislado. Se observa la equivalencia de los cuatro enlaces Si--O (\SI{1.626}{\angstrom}), aunque el poliedro presenta compresión a lo largo del eje \(c\).}
    \label{fig:zircon_td}
  \end{figure}
\end{minipage}

\vspace{0.5cm}

\begin{minipage}[c]{0.25\textwidth}
  \centering
  \begin{figure}[H]
    \includegraphics[width=\linewidth]{Imagenes/1_ZIRCON/ZIRCON_Dode.png}
    \captionsetup{justification=centering}
    \caption{Dodecaedro triangular (bisdisfenoide) \ce{ZrO8}, mostrando los dos conjuntos de distancias de enlace Zr--O: cuatro enlaces cortos (\SI{2.13}{\angstrom}) y cuatro largos (\SI{2.27}{\angstrom}).}
    \label{fig:zircon_dode_fixed}
  \end{figure}
\end{minipage}
\hfill
\begin{minipage}[c]{0.70\textwidth} % Columna de IMAGEN (Derecha)
  \setlength{\parskip}{1em} % Mantiene espacio entre párrafos
  El catión \ce{Zr^{4+}}, ubicado en el sitio de Wyckoff \(4a\) con simetría puntual \(\bar{4}m2\), presenta un número de coordinación 8, conformando un dodecaedro triangular (bisdisfenoide) \ce{ZrO8}. A diferencia del tetraedro de silicio, este poliedro exhibe dos conjuntos cristalográficamente distintos de distancias de enlace: cuatro enlaces Zr--O más cortos de \SI{2.13}{\angstrom} y cuatro más largos de \SI{2.27}{\angstrom}. Esta bimodalidad origina un índice de distorsión de la longitud de enlace de 0.0318, significativamente mayor que el valor nulo observado para el Si. La distancia media ponderada Zr--O es de \SI{2.20}{\angstrom}, con un volumen poliédrico de \SI{19.18}{\cubic\angstrom}.
\end{minipage}
\vspace{0.5cm}

\begin{minipage}[c]{0.55\textwidth} % Columna de TEXTO (Izquierda)
  \setlength{\parskip}{1em} % Mantiene espacio entre párrafos
  Estos poliedros están organizados en una red tridimensional que define la estructura del Zircón. En la Figura \ref{fig:Zircón_lattice}, se puede ver cómo los tetraedros \ce{SiO4} en amarillo comparten aristas de forma alternada e intercalada con los dodecaedros \ce{ZrO8} en azul, formando cadenas a lo largo del eje c. Estos, a su vez, comparten también aristas entre ellos.
\end{minipage}
\hfill
\begin{minipage}[c]{0.40\textwidth} % Columna de IMAGEN (Derecha)
  \centering
  \begin{figure}[H]
    \centering
    \includegraphics[width=\textwidth]{Imagenes/1_ZIRCON/POLYHEDRA/ZIRCON_pol_conex.png}
    \caption{Conectividad poliédrica en la estructura del zircón: los tetraedros \ce{SiO4} (amarillo) se conectan a los dodecaedros \ce{ZrO8} (verde) mediante vértices compartidos, generando una red tridimensional sin aristas ni caras comunes.}
    \label{fig:Zircón_lattice}
  \end{figure}
\end{minipage}
\vspace{0.5cm}

En la proyección a lo largo del eje \(a\) (Figura \ref{fig:zircon_a_poly}), se revela la alternancia ordenada de poliedros \ce{ZrO8} y \ce{SiO4} característica de esta estructura. En el plano (100), los tetraedros (amarillo) y dodecaedros (azul) se disponen formando cadenas que se extienden paralelas al eje \(c\). La transición al plano (200) evidencia el desplazamiento de medio parámetro \(a\), consistente con la operación de simetría del plano de deslizamiento \(a\) incluido en el grupo espacial \(I4_1/amd\).

\begin{figure}[H]
  \centering
  \begin{subfigure}[b]{0.45\linewidth}
    \includegraphics[width=\linewidth]{Imagenes/1_ZIRCON/POLYHEDRA/ZIRCON_a0.png}
    \caption{Distribución de los poliedros en el plano (1 0 0)}
  \end{subfigure}
  \hfill
  \begin{subfigure}[b]{0.45\linewidth}
    \includegraphics[width=\linewidth]{Imagenes/1_ZIRCON/POLYHEDRA/ZIRCON_a1.png}
    \caption{Distribución de los poliedros en el plano (2 0 0)}
  \end{subfigure}
  \caption{Proyección poliédrica del zircón a lo largo del eje \(a\): alternancia de tetraedros \ce{SiO4} (amarillo) y dodecaedros \ce{ZrO8} (verde) en los planos (100) y (200).}
  \label{fig:zircon_a_poly}
\end{figure}

En la representación del esqueleto atómico (Figura \ref{fig:zircon_a_wire}), se observa la misma disposición ordenada de los átomos que conforman los poliedros \ce{ZrO8} y \ce{SiO4}, resaltando la simetría tetragonal de la estructura del Zircón.

\begin{figure}[H]
  \centering
  \begin{subfigure}[b]{0.45\linewidth}
    \includegraphics[width=\linewidth]{Imagenes/1_ZIRCON/WIREFRAME/ZIRCON_a0.png}
    \caption{Distribución del esqueleto en el plano (1 0 0)}
  \end{subfigure}
  \hfill
  \begin{subfigure}[b]{0.45\linewidth}
    \includegraphics[width=\linewidth]{Imagenes/1_ZIRCON/WIREFRAME/ZIRCON_a1.png}
    \caption{Distribución del esqueleto en el plano (2 0 0)}
  \end{subfigure}
  \caption{Representación en modo esqueleto del zircón a lo largo del eje \(a\), mostrando las conexiones interatómicas en los planos (100) y (200).}
  \label{fig:zircon_a_wire}
\end{figure}

La proyección a lo largo del eje \(b\) muestra una disposición similar de los poliedros \ce{ZrO8} y \ce{SiO4}, con una orientación diferente (giro de 90\(^\circ\)) que resalta la simetría tetragonal de la estructura del Zircón.

La representacion de los poliedros a lo largo del eje \(c\) (Figura \ref{fig:zircon_c_poly}) revela una disposición ordenada de los poliedros \ce{ZrO8} y \ce{SiO4}, donde los tetraedros amarillos y los dodecaedros azules comparten vértices y forman una red tridimensional. Esta vista resalta la simetría tetragonal de la estructura del Zircón.

\begin{figure}[H]
  \centering
  \begin{subfigure}[b]{0.45\linewidth}
    \includegraphics[width=\linewidth]{Imagenes/1_ZIRCON/POLYHEDRA/ZIRCON_c0.png}
    \caption{Proyección de los poliedros en el plano (0 0 1)}
  \end{subfigure}
  \hfill
  \begin{subfigure}[b]{0.45\linewidth}
    \includegraphics[width=\linewidth]{Imagenes/1_ZIRCON/POLYHEDRA/ZIRCON_c1.png}
    \caption{Proyección del esqueleto en el plano (0 0 2)}
  \end{subfigure}
  \caption{Proyección poliédrica del zircón a lo largo del eje \(c\), revelando la simetría tetragonal del grupo espacial \(I4_1/amd\).}
  \label{fig:zircon_c_poly}
\end{figure}

\vspace{0.5cm}

\begin{table}[H]
  \centering
  \caption{Comparación de parámetros de distorsión entre los tetraedros \ce{SiO4} y los dodecaedros \ce{ZrO8} en la estructura del zircón.}
  \label{tab:distorsion_zircon}
  \begin{tabular}{l c c}
    \toprule
    Parámetro & \ce{SiO4} & \ce{ZrO8} \\
    \midrule
    Número de coordinación & 4 & 8 \\
    Distancia media M--O (\si{\angstrom}) & 1.626 & 2.20 \\
    Índice de distorsión (BLD) & 0.0000 & 0.0318 \\
    Varianza angular (\si{\degree\squared}) & 97.71 & --- \\
    Alargamiento cuadrático & 1.0238 & --- \\
    Volumen (\si{\cubic\angstrom}) & 2.13 & 19.18 \\
    \bottomrule
  \end{tabular}
\end{table}




%! ============================================================================
%! RECOMENDACIONES DE MEJORA PARA LA SECCIÓN 3.2 (XENÓTIMO)
%! Estrategia: De la descripción a la comparación
%! Fuente: README.md del repositorio
%! ============================================================================

%! ----------------------------------------------------------------------------
%! 1. PÁRRAFO INTRODUCTORIO: ESTABLECER LA RELACIÓN
%! ----------------------------------------------------------------------------
%! En lugar de repetir la descripción del grupo espacial, enfatiza la 
%! SUSTITUCIÓN ISOVALENTE y sus consecuencias.
%!
%! TEXTO SUGERIDO:
%! "El Xenótimo cristaliza en el mismo grupo espacial que el Zircón (I4_1/amd), 
%! constituyendo un ejemplo de isoestructuralismo derivado de la sustitución 
%! heterovalente acoplada Zr^{4+} + Si^{4+} -> Y^{3+} + P^{5+}. Esta sustitución 
%! mantiene la neutralidad de carga pero introduce diferencias significativas en 
%! las distancias de enlace y la geometría poliédrica, consecuencia directa de 
%! los distintos radios iónicos: r(Y^{3+}) = 1.019 Å vs. r(Zr^{4+}) = 0.84 Å 
%! en coordinación 8."
%! ----------------------------------------------------------------------------

%! ----------------------------------------------------------------------------
%! 2. TABLA COMPARATIVA (ELEMENTO CENTRAL)
%! ----------------------------------------------------------------------------
%! En lugar de repetir tablas individuales, crea una TABLA COMPARATIVA que 
%! destaque las diferencias entre Zircón y Xenótimo.
%!
%! INCLUIR:
%! - Parámetros de celda: a=b, c, V con diferencias porcentuales
%! - Tetraedro TO4: distancia T-O, varianza angular, volumen
%! - Dodecaedro MO8: distancias M-O corta/larga, BLD, volumen
%!
%! VER: tab:comparacion_zircon_xenotimo en README.md para plantilla completa
%! ----------------------------------------------------------------------------

%! ----------------------------------------------------------------------------
%! 3. TEXTO ANALÍTICO (NO DESCRIPTIVO)
%! ----------------------------------------------------------------------------
%! Después de la tabla, INTERPRETA los datos en lugar de repetir descripciones.
%!
%! PUNTOS CLAVE A DESTACAR:
%! - Expansión anisotrópica de la celda: parámetro a se expande (+4.1%), 
%!   c permanece casi invariante (+0.5%)
%! - Contracción del tetraedro: PO4 más pequeño que SiO4 (-13.6% volumen) 
%!   por mayor carga de P^{5+}
%! - Regularización del dodecaedro: BLD de YO8 es ~mitad del ZrO8
%! ----------------------------------------------------------------------------

%! ----------------------------------------------------------------------------
%! 4. FIGURAS: SELECCIÓN ESTRATÉGICA
%! ----------------------------------------------------------------------------
%! NO REPITAS todas las proyecciones. Incluye solo las que aporten información 
%! COMPARATIVA:
%!
%! ✅ SÍ INCLUIR:
%! - Proyecciones a, b, c de la celda unitaria (con caption comparativo)
%! - NUEVA figura comparativa lado a lado (Zircón vs Xenótimo, eje c)
%!
%! ⚠️ OPCIONAL:
%! - Tetraedro PO4 individual (solo si muestras diferencia visual vs SiO4)
%! - Dodecaedro YO8 individual (ídem)
%!
%! ❌ REDUCIR:
%! - Proyecciones de poliedros para TODOS los ejes (incluir solo UNA 
%!   proyección representativa)
%! ----------------------------------------------------------------------------

%! ----------------------------------------------------------------------------
%! 5. FIGURA COMPARATIVA SUGERIDA (NUEVA)
%! ----------------------------------------------------------------------------
%! Crear figura con subfigures lado a lado:
%! - (a) Zircón: proyección [001] desde ZIRCON_c0.png
%! - (b) Xenótimo: proyección [001] desde XENOTIMO_c.png
%!
%! CAPTION SUGERIDO:
%! "Comparación de la proyección a lo largo del eje c entre Zircón y Xenótimo. 
%! A pesar del isoestructuralismo, se aprecia la expansión de los dodecaedros 
%! YO8 (verde) respecto a los ZrO8."
%! ----------------------------------------------------------------------------

%! ----------------------------------------------------------------------------
%! 6. CONEXIÓN CON APLICACIONES (CIERRE DE SECCIÓN)
%! ----------------------------------------------------------------------------
%! Termina conectando las diferencias estructurales con las propiedades:
%!
%! TEXTO SUGERIDO:
%! "Estas diferencias estructurales tienen implicaciones directas en las 
%! propiedades físicas y aplicaciones de ambos materiales. La mayor regularidad 
%! del poliedro YO8 en el Xenótimo, combinada con la flexibilidad del sitio 
%! para acomodar otros lantánidos pesados (Tb-Lu), fundamenta su uso como 
%! matriz para la inmovilización de actínidos en residuos nucleares. 
%! La contracción lantánida permite una modulación sistemática de los 
%! parámetros de celda a lo largo de la serie LnPO4."
%! ----------------------------------------------------------------------------

%! ============================================================================
%! RESUMEN DE LA ESTRATEGIA
%! ============================================================================
%!
%! | Elemento              | Sec. 3.1 (Zircón)      | Sec. 3.2 (Xenótimo)      |
%! |-----------------------|------------------------|--------------------------|
%! | Enfoque               | Descriptivo completo   | COMPARATIVO              |
%! | Grupo espacial        | Explicación detallada  | Ref. breve ("isoestruct.")|
%! | Tabla de coordenadas  | Completa               | Mantener (Z diferente)   |
%! | Tabla de distorsión   | Individual             | COMPARATIVA con Zircón   |
%! | Figuras de poliedros  | Todas                  | Reducir + fig comparativa|
%! | Texto                 | Descripción            | Análisis de diferencias  |
%!
%! ============================================================================

\section{Xenótimo (\texorpdfstring{\ce{LnPO4}}{LnPO4}, Ln = Y, Tb-Lu)}
Debido a la contracción lantánida, los lantánidos pesados (HREE) como el Y, Tb-Lu poseen un radio iónico similar al Zr\(^{4+}\). Esto permite que el \ce{YPO4} cristalice en la misma estructura tetragonal que el Zircón, manteniendo el grupo espacial \(I4_1/amd\). A pesar de ser isoestructurales \cite{chong_synthesis_2024}, la sustitución de Si\(^{4+}\) por P\(^{5+}\) y Zr\(^{4+}\) por Y\(^{3+}\) provoca una ligera expansión de la celda (volumen 286 vs 260 \(\text{\AA}^3\)), aunque el poliedro de coordinación \ce{YO8} conserva la geometría de dodecaedro del \ce{ZrO8}.

Los parámetros cristalográficos del Xenótimo \ce{YPO4} se obtienen de ``Mindat.org'' \footnote{\url{https://www.mindat.org/min-4333.html}} \cite{ni1995crystal}.

La estructura del Xenótimo, isoestructural con el Zircón, se caracteriza por un sistema cristalino tetragonal con grupo espacial \(I4_1/amd\) (No. 141). I indica un átomo centrado en la celda, \(4_1\) indica un eje de \(90^\circ\) de rotación con una traslación de 1/4 de la altura de la celda unitaria, \(a\) indica un plano de simetría perpendicular al eje principal, \(m\) indica un plano de simetría y \(d\) indica un plano de simetría diagonal. El grupo puntual \(4/mmm\): \(4/m\) indica un eje de rotación de \(90^\circ\) con un plano de simetría horizontal perpendicular al eje principal \(c\). La segunda \(m\) indica un plano de simetría vertical al eje principal que pasa por los ejes \(a\) y \(b\). La tercera \(m\) indica un plano de simetría vertical al eje principal que pasa por las diagonales del plano \(ab\). Con \(Z=4\) unidades de fórmula por celda unitaria.

Las coordenadas atómicas, factores de ocupación y parámetros térmicos isotrópicos para la estructura del Xenótimo se presentan en la Tabla \ref{tab:coord_xenotimo}.

\begin{table}[H]
  \centering
  \caption{Parámetros cristalográficos del xenótimo (\ce{YPO4}): coordenadas fraccionarias, ocupación, desplazamientos térmicos, posiciones de Wyckoff y simetría puntual.}
  \label{tab:coord_xenotimo}
  \vspace{0.1cm}
  \begin{tabular}{l c c c c c c c}
    \toprule
    Átomo & \(x\)  & \(y\)  & \(z\)  & Ocup & \(U\) & Pos. Wyckoff & Sim.          \\
    \midrule
    Y     & 0.0000 & 0.7500 & 0.1250 & 1    & 0.004 & \(4a\)       & \(\bar{4}m2\) \\
    P     & 0.0000 & 0.2500 & 0.3750 & 1    & 0.005 & \(4b\)       & \(\bar{4}m2\) \\
    O     & 0.0000 & 0.0764 & 0.2175 & 1    & 0.008 & \(16h\)      & \(.m.\)       \\
    \bottomrule
  \end{tabular}
\end{table}

Los parámetros de la celda unitaria son \(a = b \neq c\) con \(a=b=6.8947, c = 6.0276\) \si{\angstrom} y los ángulos \(\alpha = \beta = \gamma = 90^\circ\). La presencia de átomos de Y en vez de Zr provoca un aumento en el volumen de la celda unitaria respecto al Zircón, que en este caso es \(286.533\) \si{\cubic\angstrom}. Cada celda unitaria contiene 4 unidades de fórmula (\(Z=4\)); por lo tanto, la fórmula empírica es \ce{Y4P4O16}. La celda unitaria se muestra en las proyecciones a lo largo de los ejes \(a\), \(b\) y \(c\) en la Figura \ref{fig:xenotimo_uc}.

\begin{figure}[H]
  \centering
  \begin{subfigure}[b]{0.3\linewidth}
    \includegraphics[width=\linewidth]{Imagenes/2_XENOTIMO/ATOMS/XENOTIMO_a.png}
    \caption{Eje \(a\)}
    \label{fig:xenotimo_atoms_a}
  \end{subfigure}
  \hfill
  \begin{subfigure}[b]{0.3\linewidth}
    \includegraphics[width=\linewidth]{Imagenes/2_XENOTIMO/ATOMS/XENOTIMO_b.png}
    \caption{Eje \(b\)}
    \label{fig:xenotimo_atoms_b}
  \end{subfigure}
  \hfill
  \begin{subfigure}[b]{0.3\linewidth}
    \includegraphics[width=\linewidth]{Imagenes/2_XENOTIMO/ATOMS/XENOTIMO_c.png}
    \caption{Eje \(c\)}
    \label{fig:xenotimo_atoms_c}
  \end{subfigure}
  \caption{Proyecciones de la celdilla unidad del xenótimo a lo largo de los ejes cristalográficos principales. Código de colores: Y (verde claro), P (naranja), O (rojo).}
  \label{fig:xenotimo_uc}
\end{figure}

\vspace{0.5cm}

\begin{minipage}[c]{0.70\textwidth}
  \setlength{\parskip}{1em} % Mantiene espacio entre párrafos
  En cuanto al entorno de coordinación del fósforo, este forma un tetraedro \ce{PO4} con cuatro longitudes de enlace idénticas de \SI{1.533}{\angstrom}, lo que resulta en un índice de distorsión de la longitud de enlace de 0.0000. Aunque mantiene la simetría local del sitio \(4b\) (\(\bar{4}m2\)) al igual que el silicio en el Zircón, el tetraedro de fosfato es notablemente menos distorsionado angularmente: presenta una varianza del ángulo de enlace de \SI{21.48}{\degree\squared} (frente a los \SI{97.71}{\degree\squared} del silicato) y un volumen más reducido de \SI{1.84}{\cubic\angstrom}, consistente con el menor radio iónico del \ce{P^{5+}}.
\end{minipage}% <--- ¡IMPORTANTE: No borres este símbolo de porcentaje!
\hfill
\begin{minipage}[c]{0.25\textwidth} % Columna de IMAGEN (Derecha)
  \centering
  \begin{figure}[H]
    \includegraphics[width=\linewidth]{Imagenes/2_XENOTIMO/XENOTIMO_Td.png}
    \captionsetup{justification=centering}
    \caption{Tetraedro \ce{PO4} del xenótimo con cuatro enlaces P--O equivalentes de \SI{1.533}{\angstrom}, reflejando la simetría \(\bar{4}m2\) del sitio \(4b\).}
    \label{fig:xenotimo_td_fixed}
  \end{figure}
\end{minipage}

\vspace{0.5cm}

\begin{minipage}[C]{0.3\textwidth} % Columna de TEXTO (Izquierda)
  \setlength{\parskip}{1em} % Mantiene espacio entre párrafos
  \centering
  \begin{figure}[H]
    \includegraphics[width=\linewidth]{Imagenes/2_XENOTIMO/XENOTIMO_Dode.png}
    \caption{Bisdisfenoide \ce{YO8} del xenótimo, mostrando dos conjuntos de distancias Y--O: \SI{2.31}{\angstrom} (cortas) y \SI{2.38}{\angstrom} (largas).}
    \label{fig:xenotimo_dode_fixed}
  \end{figure}
\end{minipage}
\hfill
\begin{minipage}[c]{0.65\textwidth}
  \setlength{\parskip}{1em} % Mantiene espacio entre párrafos
  Por otro lado, el catión \ce{Y^{3+}} se encuentra en una coordinación 8 formando un dodecaedro \ce{YO8}. Este poliedro exhibe dos conjuntos de distancias de enlace, con cuatro enlaces más cortos de \SI{2.311}{\angstrom} y cuatro más largos de \SI{2.391}{\angstrom}, resultando en una distancia media de \SI{2.351}{\angstrom}. El índice de distorsión de la longitud de enlace es de 0.0169, un valor inferior al observado para el \ce{Zr^{4+}} en la estructura del Zircón (0.0318), lo que indica un entorno ligeramente más regular para el itrio. Debido al mayor tamaño del catión trivalente, el volumen del poliedro es significativamente mayor, alcanzando los \SI{23.30}{\cubic\angstrom} (ver Figuras \ref{fig:xenotimo_big_a} y \ref{fig:xenotimo_big_c}).
\end{minipage}

\vspace{0.5cm}

En esta estructura, los iones \ce{Y^3+} están coordinados por ocho átomos de oxígeno, formando poliedros \ce{YO8} que se disponen alrededor de los tetraedros \ce{PO4}. Los tetraedros \ce{PO4} están orientados de manera que sus vértices apuntan hacia los centros de los poliedros \ce{YO8}, creando una red tridimensional interconectada. Esta disposición resulta en una estructura robusta y estable, característica del Xenótimo. Las representaciones de los poliedros \ce{YO8} y \ce{PO4}, así como del esqueleto de la estructura, se muestran en las Figuras \ref{fig:xenotimo_big_a}, \ref{fig:xenotimo_big_b} y \ref{fig:xenotimo_big_c}.

\begin{figure}[H]
  \centering
  \begin{subfigure}[b]{0.45\linewidth}
    \includegraphics[width=\linewidth]{Imagenes/2_XENOTIMO/POLYHEDRA/XENOTIMO_a.png}
    \caption{Proyección de los poliedros}
  \end{subfigure}
  \hfill
  \begin{subfigure}[b]{0.45\linewidth}
    \includegraphics[width=\linewidth]{Imagenes/2_XENOTIMO/WIREFRAME/XENOTIMO_a.png}
    \caption{Proyección del esqueleto}
  \end{subfigure}
  \caption{Proyección del xenótimo a lo largo del eje \(a\): (a) representación poliédrica y (b) modo esqueleto.}
  \label{fig:xenotimo_big_a}
\end{figure}

\begin{figure}[H]
  \centering
  \begin{subfigure}[b]{0.45\linewidth}
    \includegraphics[width=\linewidth]{Imagenes/2_XENOTIMO/POLYHEDRA/XENOTIMO_b.png}
    \caption{Proyección a lo largo del eje \(b\).}
  \end{subfigure}
  \hfill
  \begin{subfigure}[b]{0.45\linewidth}
    \includegraphics[width=\linewidth]{Imagenes/2_XENOTIMO/WIREFRAME/XENOTIMO_b.png}
    \caption{Proyección a lo largo del eje \(b\).}
  \end{subfigure}
  \caption{Proyección del xenótimo a lo largo del eje \(b\): (a) representación poliédrica y (b) modo esqueleto.}
  \label{fig:xenotimo_big_b}
\end{figure}


\begin{figure}[H]
  \centering
  \begin{subfigure}[b]{0.45\linewidth}
    \includegraphics[width=\linewidth]{Imagenes/2_XENOTIMO/POLYHEDRA/XENOTIMO_c.png}
    \caption{Proyección de los poliedros}
  \end{subfigure}
  \hfill
  \begin{subfigure}[b]{0.45\linewidth}
    \includegraphics[width=\linewidth]{Imagenes/2_XENOTIMO/WIREFRAME/XENOTIMO_c.png}
    \caption{Proyección del esqueleto}
  \end{subfigure}
  \caption{Proyección del xenótimo a lo largo del eje \(c\), evidenciando la simetría tetragonal isoestructural con el zircón.}
  \label{fig:xenotimo_big_c}
\end{figure}


\begin{figure}[H]
  \centering
  \includegraphics[width=0.8\textwidth]{Imagenes/2_XENOTIMO/XENOTIMO.png}
  \caption{Vista tridimensional del xenótimo mostrando la disposición espacial de los poliedros \ce{YO8} (verde) y \ce{PO4} (amarillo).}
  \label{fig:xenotimo_pol_perspective}
\end{figure}

\section{Churchita (\texorpdfstring{\ce{LnPO4*nH2O}}{LnPO4*nH2O}, Ln= Y, Tb-Lu)}
En condiciones de síntesis a baja temperatura (precipitación), los HREE no forman directamente la fase anhidra (Xenótimo), sino una fase metaestable hidratada: la Churchita \cite{chong_synthesis_2024}. La incorporación de moléculas de agua en la red impide la simetría tetragonal, reduciendo el sistema a monoclínico (\(C2/c\)). Su estructura es laminar debido a la presencia de moléculas de agua en su red cristalina. Es la fase precursora del Xenótimo, estable solo a bajas temperaturas (<200-300 °C) \cite{rafiuddin_review_2022,rafiuddin_structural_2022}.

Los parámetros cristalográficos para la representación de la Churchita \ce{YPO4*2H2O} se obtienen de ``Crystallography Open Database'' \footnote{\url{https://www.mindat.org/min-1047.html}} \cite{kohlmann1994structure} .

La estructura de la Churchita se caracteriza por un sistema cristalino monoclínico con grupo espacial \textit{C2/c} (No. 15), lo que indica una celda de tipo centrada en el cuerpo (\(C\)). \(2\) indica un eje de rotación de \(180^\circ\) y \(c\) indica un plano de simetría diagonal. El grupo puntual \(2/m\): \(2\) indica un eje de rotación de \(180^\circ\) y \(m\) indica un plano de simetría perpendicular al eje principal \(b\).

Las coordenadas atómicas, factores de ocupación y parámetros térmicos isotrópicos para la estructura de la Churchita se presentan en la Tabla \ref{tab:coord_churchita}.

\begin{table}[H]
  \centering
  \caption{Parámetros cristalográficos de la churchita (\ce{YPO4*2H2O}): coordenadas fraccionarias, ocupación, desplazamientos térmicos, posiciones de Wyckoff y simetría puntual.}
  \label{tab:coord_churchita}
  \vspace{0.1cm}
  \begin{tabular}{l c c c c c c c}
    \toprule
    Átomo & \(x\)  & \(y\)  & \(z\)  & Ocup & \(U\) & Pos. Wyckoff & Sim.  \\
    \midrule
    Y     & 0.2500 & 0.8289 & 0.0000 & 1    & 0.007 & \(4e\)       & \(2\) \\
    P     & 0.2500 & 0.3307 & 0.0000 & 1    & 0.014 & \(4e\)       & \(2\) \\
    O1    & 0.3020 & 0.3857 & 0.2240 & 1    & 0.022 & \(8f\)       & \(1\) \\
    O2    & 0.5060 & 0.2714 & 0.0840 & 1    & 0.018 & \(8f\)       & \(1\) \\
    Ow    & 0.6300 & 0.0680 & 0.2180 & 1    & 0.021 & \(8f\)       & \(1\) \\
    \bottomrule
  \end{tabular}
\end{table}

Los parámetros de la celda unitaria son \(a \neq b \neq c\), con \(a= 5.5780, b= 15.0060, c= 6.2750\) \si{\angstrom} y los ángulos \(\alpha = \gamma = 90^\circ\), \(\beta = 117.83^\circ\). El volumen de la celda unitaria es \(464.488323\) \si{\cubic\angstrom}. Cada celda unitaria contiene 4 unidades de fórmula (\(Z=4\)); por lo tanto, la fórmula empírica es \ce{Y4P4O16*8H2O}. La celda unitaria se muestra en las proyecciones a lo largo de los ejes \(a\), \(b\) y \(c\) en la Figura \ref{fig:churchita_uc}.

\begin{figure}[H]
  \centering
  \begin{minipage}[c]{0.5\linewidth}
    \centering
    \begin{subfigure}[b]{\linewidth}
      \centering
      \includegraphics[width=\linewidth]{Imagenes/3_CHURCHITA/ATOMS/CHURCHITA_a.png}
      \caption{Eje \(a\)}
      \label{fig:churchita_atoms_a}
    \end{subfigure}
    \hfill
    \begin{subfigure}[b]{0.8\linewidth}
      \centering
      \includegraphics[width=\linewidth]{Imagenes/3_CHURCHITA/ATOMS/CHURCHITA_b.png}
      \caption{Eje \(b\)}
      \label{fig:churchita_atoms_b}
    \end{subfigure}
  \end{minipage}
  \hfill
  \begin{minipage}[c]{0.25\linewidth} % Ocupa el 45% del ancho
    \centering
    \begin{subfigure}[b]{\linewidth}
      \centering
      \includegraphics[width=\linewidth]{Imagenes/3_CHURCHITA/ATOMS/CHURCHITA_c.png}
      \caption{Eje \(c\)}
      \label{fig:churchita_atoms_c}
    \end{subfigure}
  \end{minipage}
  \caption{Proyecciones de la celdilla unidad de la churchita. Código de colores: Y (verde), P (naranja), O (rojo), \ce{H2O} (azul claro).}
  \label{fig:churchita_uc}
\end{figure}

\vspace{0.5cm}

\begin{minipage}[c]{0.70\textwidth} % Columna de TEXTO
  \setlength{\parskip}{1em}
  El entorno de coordinación del fósforo conforma un tetraedro \ce{PO4} que, a diferencia de la alta simetría observada en el Xenótimo, presenta una ligera distorsión. Se distinguen dos conjuntos de distancias de enlace P--O: dos enlaces P--O1 de \SI{1.536}{\angstrom} y dos enlaces P--O2 ligeramente más largos de \SI{1.551}{\angstrom}. Esta leve irregularidad se traduce en un índice de distorsión de la longitud de enlace de 0.0048 y una varianza angular de \SI{18.96}{\degree\squared}. El volumen del tetraedro es de \SI{1.88}{\cubic\angstrom}, manteniendo la rigidez característica del grupo fosfato con una distancia media de enlace de \SI{1.544}{\angstrom}.
\end{minipage}
\hfill
\begin{minipage}[c]{0.25\textwidth} % Columna de IMAGEN (Opcional)
  \centering
  \begin{figure}[H]
    \includegraphics[width=\linewidth]{Imagenes/3_CHURCHITA/CHURCHITA_Td.png}
    \captionsetup{justification=centering}
    \caption{Tetraedro \ce{PO4} de la churchita con dos conjuntos de distancias P--O, reflejando la menor simetría del sistema monoclínico \(C2/c\).}
    \label{fig:churchita_td}
  \end{figure}
\end{minipage}

\vspace{0.5cm}

\begin{minipage}[c]{0.3\textwidth} % Columna de IMAGEN (Izquierda)
  \centering
  \begin{figure}[H]
    \includegraphics[width=\linewidth]{Imagenes/3_CHURCHITA/CHURCHITA_Dode.png}
    \captionsetup{justification=centering}
    \caption{Poliedro de coordinación \ce{YO6(H2O)2} de la churchita, integrando seis oxígenos de fosfato y dos moléculas de agua estructural.}
    \label{fig:churchita_dode}
  \end{figure}
\end{minipage}%
\hfill
\begin{minipage}[c]{0.65\textwidth} % Texto a todo lo ancho
  \setlength{\parskip}{1em}
  Por su parte, el catión itrio (\ce{Y^{3+}}) presenta un número de coordinación 8, conformando un poliedro de geometría compleja que integra tanto átomos de oxígeno de los grupos fosfato como moléculas de agua estructural. El entorno de coordinación es notablemente heterogéneo, exhibiendo cuatro pares de distancias de enlace diferenciadas: los enlaces más cortos corresponden a Y--O2 (\SI{2.251}{\angstrom}), seguidos por la interacción con las moléculas de agua Y--Wat3 (\SI{2.360}{\angstrom}), y finalmente los enlaces más largos Y--O1 (\SI{2.435}{\angstrom}) y el segundo conjunto de Y--O2 (\SI{2.469}{\angstrom}). Esta variabilidad en las longitudes de enlace genera un índice de distorsión de 0.0308, superior al observado en la fase anhidra (Xenótimo), y una distancia media de enlace de \SI{2.379}{\angstrom}. El volumen de este poliedro hidratado es de \SI{23.81}{\cubic\angstrom}. La presencia de las moléculas de agua en la esfera de coordinación es el factor clave que diferencia estructuralmente a la Churchita del Xenótimo, permitiendo la formación de una red de enlaces de hidrógeno que estabiliza la estructura monoclínica.
\end{minipage}

\vspace{0.5cm}

En esta estructura, los iones \ce{Y^3+} están coordinados por ocho átomos de oxígeno y moléculas de agua, formando poliedros \ce{YPO4*nH2O} que se disponen alrededor de los tetraedros \ce{PO4}.

Las representaciones de los poliedros \ce{YPO4*nH2O} y \ce{PO4}, así como del esqueleto de la estructura, se muestran en las Figuras \ref{fig:churchita_big_a}, \ref{fig:churchita_big_b} y \ref{fig:churchita_big_c}.


\begin{figure}[H]
  \centering
  \begin{subfigure}[b]{0.45\linewidth}
    \includegraphics[width=\linewidth]{Imagenes/3_CHURCHITA/POLYHEDRA/CHURCHITA_a.png}
    \caption{Proyección de los poliedros}
  \end{subfigure}
  \hfill
  \begin{subfigure}[b]{0.45\linewidth}
    \includegraphics[width=\linewidth]{Imagenes/3_CHURCHITA/WIREFRAME/CHURCHITA_a.png}
    \caption{Proyección del esqueleto}
  \end{subfigure}
  \caption{Proyección de la churchita a lo largo del eje \(a\): (a) representación poliédrica y (b) modo esqueleto.}
  \label{fig:churchita_big_a}
\end{figure}

\begin{figure}[H]
  \centering
  \begin{subfigure}[b]{0.45\linewidth}
    \includegraphics[width=\linewidth]{Imagenes/3_CHURCHITA/POLYHEDRA/CHURCHITA_b.png}
    \caption{Proyección a lo largo del eje \(b\).}
  \end{subfigure}
  \hfill
  \begin{subfigure}[b]{0.45\linewidth}
    \includegraphics[width=\linewidth]{Imagenes/3_CHURCHITA/WIREFRAME/CHURCHITA_b.png}
    \caption{Proyección a lo largo del eje \(b\).}
  \end{subfigure}
  \caption{Proyección de la churchita a lo largo del eje \(b\), perpendicular a las capas estructurales del sistema monoclínico.}
  \label{fig:churchita_big_b}
\end{figure}

\begin{figure}[H]
  \centering
  \begin{subfigure}[b]{0.45\linewidth}
    \includegraphics[width=\linewidth]{Imagenes/3_CHURCHITA/POLYHEDRA/CHURCHITA_c.png}
    \caption{Proyección de los poliedros}
  \end{subfigure}
  \hfill
  \begin{subfigure}[b]{0.45\linewidth}
    \includegraphics[width=\linewidth]{Imagenes/3_CHURCHITA/WIREFRAME/CHURCHITA_c.png}
    \caption{Proyección del esqueleto}
  \end{subfigure}
  \caption{Proyección de la churchita a lo largo del eje \(c\): (a) representación poliédrica y (b) modo esqueleto.}
  \label{fig:churchita_big_c}
\end{figure}

\begin{figure}[H]
  \centering
  \includegraphics[width=0.8\textwidth]{Imagenes/3_CHURCHITA/CHURCHITA.png}
  \caption{Vista tridimensional de la churchita mostrando los poliedros \ce{YO6(H2O)2} (verde) y \ce{PO4} (amarillo) característicos de la fase hidratada.}
  \label{fig:churchita_pol_perspective}
\end{figure}

\section{Monacita (\texorpdfstring{\ce{LnPO4}}{LnPO4}, Ln = La-Gd)}
A medida que recorremos la serie lantánida hacia la izquierda (LREE: La-Gd), el radio iónico aumenta significativamente. El catión \ce{Ce^3+} es demasiado grande para acomodarse en el hueco dodecaédrico de coordinación 8 del Xenótimo/Zircón. Como consecuencia, la estructura sufre una transición morfotrópica hacia la fase Monacita (monoclínica, \(P2_1/n\)), permitiendo un aumento en el número de coordinación a 9.

Los parámetros cristalográficos para la representación de la Monacita \ce{CePO4} se obtienen de ``Crystallography Open Database'' \footnote{\url{https://www.crystallography.net/cod/9016405.html}} \cite{mooney1948crystal}.

La estructura de la Monacita se caracteriza por un sistema cristalino monoclínico con grupo espacial \(P2_1/n\) (No. 14), lo que indica una celda de tipo primitiva (\(P\)). \(2_1\) indica un eje helicoidal de \(180^\circ\) con una traslación de 1/2 de la altura de la celda unitaria y \(n\) indica un plano de simetría diagonal. El grupo puntual \(2/m\): \(2\) indica un eje de rotación de \(180^\circ\) y \(m\) indica un plano de simetría perpendicular al eje principal \(b\).

Las coordenadas atómicas, factores de ocupación y parámetros térmicos isotrópicos para la estructura de la Monacita se presentan en la Tabla \ref{tab:coord_monacita}.

\begin{table}[H]
  \centering
  \caption{Parámetros cristalográficos de la monacita (\ce{CePO4}): coordenadas fraccionarias, ocupación, desplazamientos térmicos, posiciones de Wyckoff y simetría puntual.}
  \label{tab:coord_monacita}
  \vspace{0.1cm}
  \begin{tabular}{l c c c c c c c}
    \toprule
    Átomo & \(x\) & \(y\) & \(z\) & Ocup & \(U\) & Pos. Wyckoff & Sim.  \\
    \midrule
    Ce    & 0.292 & 0.156 & 0.083 & 1    & 0.000 & \(4e\)       & \(1\) \\
    P     & 0.292 & 0.156 & 0.581 & 1    & 0.000 & \(4e\)       & \(1\) \\
    O1    & 0.211 & 0.990 & 0.423 & 1    & 0.000 & \(4e\)       & \(1\) \\
    O2    & 0.374 & 0.323 & 0.464 & 1    & 0.000 & \(4e\)       & \(1\) \\
    O3    & 0.467 & 0.086 & 0.765 & 1    & 0.000 & \(4e\)       & \(1\) \\
    O4    & 0.116 & 0.235 & 0.680 & 1    & 0.000 & \(4e\)       & \(1\) \\
    \bottomrule
  \end{tabular}
\end{table}

Los parámetros de la celda unitaria son \(a =6.76, b= 7.00, c= 6.44\) \si{\angstrom} y los ángulos \(\alpha = \gamma = 90^\circ\), \(\beta \neq 90^\circ\). El volumen de la celda unitaria es \(296.158629\) \si{\cubic\angstrom}. Cada celda unitaria contiene 4 unidades de fórmula (\(Z=4\)); por lo tanto, la fórmula empírica es \ce{Ce4P4O16}. La celda unitaria se muestra en las proyecciones a lo largo de los ejes \(a\), \(b\) y \(c\) en la Figura \ref{fig:monacita_uc}.

\begin{figure}[H]
  \centering
  \begin{subfigure}[b]{0.3\linewidth}
    \includegraphics[width=\linewidth]{Imagenes/4_MONACITA/ATOMS/MONACITA_a.png}
    \caption{Eje \(a\)}
    \label{fig:monacita_atoms_a}
  \end{subfigure}
  \hfill
  \begin{subfigure}[b]{0.3\linewidth}
    \includegraphics[width=\linewidth]{Imagenes/4_MONACITA/ATOMS/MONACITA_b.png}
    \caption{Eje \(b\)}
    \label{fig:monacita_atoms_b}
  \end{subfigure}
  \hfill
  \begin{subfigure}[b]{0.3\linewidth}
    \includegraphics[width=\linewidth]{Imagenes/4_MONACITA/ATOMS/MONACITA_c.png}
    \caption{Eje \(c\)}
    \label{fig:monacita_atoms_c}
  \end{subfigure}
  \caption{Proyecciones de la celdilla unidad de la monacita. Código de colores: Ce (amarillo), P (naranja), O (rojo).}
  \label{fig:monacita_uc}
\end{figure}

\vspace{0.5cm}

\begin{minipage}[c]{0.70\textwidth} % Columna de TEXTO (Izquierda)
  \setlength{\parskip}{1em}
  El entorno de coordinación del fósforo conforma un tetraedro \ce{PO4} extremadamente regular y rígido. Las distancias de enlace P--O son notablemente homogéneas, con un valor promedio de \SI{1.56}{\angstrom} y una varianza mínima. Esta uniformidad se refleja en un índice de distorsión de longitud de enlace muy bajo (0.0061).

  El análisis geométrico confirma la idealidad del poliedro, presentando una elongación cuadrática de 1.0004 y un número de coordinación efectivo de 3.99, prácticamente idéntico al valor teórico de 4. El volumen calculado para este tetraedro es de \SI{1.95}{\cubic\angstrom}, actuando como la unidad estructural estable de la red.
\end{minipage}%
\hfill
\begin{minipage}[c]{0.25\textwidth} % Columna de IMAGEN (Derecha)
  \centering
  \begin{figure}[H]
    % REEMPLAZA CON EL NOMBRE DE TU ARCHIVO REAL
    \includegraphics[width=\linewidth]{Imagenes/4_MONACITA/MONACITA_Td.png}
    \captionsetup{justification=centering}
    \caption{Tetraedro \ce{PO4} de la monacita con distancia media P--O de \SI{1.53}{\angstrom} y distorsión mínima (BLD $\approx$ 0).}
    \label{fig:monacita_td} % Puedes cambiar la etiqueta si quieres
  \end{figure}
\end{minipage}

\vspace{0.5cm}

\begin{minipage}[c]{0.25\textwidth} % Columna de IMAGEN (Izquierda)
  \centering
  \begin{figure}[H]
    \includegraphics[width=\linewidth]{Imagenes/4_MONACITA/MONACITA_ene.png}
    \captionsetup{justification=centering}
    \caption{Eneacedro (poliedro de 9 vértices) \ce{CeO9} de la monacita con distancias Ce--O entre \SI{2.26}{\angstrom} y \SI{2.75}{\angstrom}, evidenciando alta distorsión.}
    \label{fig:cerio_poly}
  \end{figure}
\end{minipage}%
\hfill
\begin{minipage}[c]{0.70\textwidth} % Columna de TEXTO (Derecha)
  \setlength{\parskip}{1em}
  El entorno de coordinación del cerio conforma un poliedro de nueve vértices (eneacedro) significativamente distorsionado, con un volumen calculado de \SI{31.85}{\cubic\angstrom}. A diferencia de la rigidez del fosfato, este sitio presenta un índice de distorsión de longitud de enlace de 0.05599.

  Esta irregularidad se manifiesta claramente en la dispersión de las distancias de enlace Ce--O, que varían desde \SI{2.26}{\angstrom} para el enlace más corto hasta \SI{2.75}{\angstrom} para el más largo, con un promedio de \SI{2.55}{\angstrom}. Dicha variabilidad resulta en un número de coordinación efectivo de 6.76, lo que indica que la esfera de coordinación real está más comprimida que la geométrica.
\end{minipage}

\vspace{0.5cm}

En esta estructura, los iones \ce{Ln^3+} están coordinados por nueve átomos de oxígeno, formando poliedros \ce{LnO9} que se disponen alrededor de los tetraedros \ce{PO4}.

Las representaciones de los poliedros \ce{LnO9} y \ce{PO4}, así como del esqueleto de la estructura, se muestran en las Figuras \ref{fig:monacita_big_a}, \ref{fig:monacita_big_b} y \ref{fig:monacita_big_c}.

\begin{figure}[H]
  \centering
  \begin{subfigure}[b]{0.45\linewidth}
    \includegraphics[width=\linewidth]{Imagenes/4_MONACITA/POLYHEDRA/MONACITA_a.png}
    \caption{Proyección de los poliedros}
  \end{subfigure}
  \hfill
  \begin{subfigure}[b]{0.45\linewidth}
    \includegraphics[width=\linewidth]{Imagenes/4_MONACITA/WIREFRAME/MONACITA_a.png}
    \caption{Proyección del esqueleto}
  \end{subfigure}
  \caption{Proyección de la monacita a lo largo del eje \(a\): (a) representación poliédrica y (b) modo esqueleto.}
  \label{fig:monacita_big_a}
\end{figure}

\begin{figure}[H]
  \centering
  \begin{subfigure}[b]{0.45\linewidth}
    \includegraphics[width=\linewidth]{Imagenes/4_MONACITA/POLYHEDRA/MONACITA_b.png}
    \caption{Proyección a lo largo del eje \(b\).}
  \end{subfigure}
  \hfill
  \begin{subfigure}[b]{0.45\linewidth}
    \includegraphics[width=\linewidth]{Imagenes/4_MONACITA/WIREFRAME/MONACITA_b.png}
    \caption{Proyección a lo largo del eje \(b\).}
  \end{subfigure}
  \caption{Proyección de la monacita a lo largo del eje \(b\): (a) representación poliédrica y (b) modo esqueleto.}
  \label{fig:monacita_big_b}
\end{figure}

\begin{figure}[H]
  \centering
  \begin{subfigure}[b]{0.45\linewidth}
    \includegraphics[width=\linewidth]{Imagenes/4_MONACITA/POLYHEDRA/MONACITA_c.png}
    \caption{Proyección de los poliedros}
  \end{subfigure}
  \hfill
  \begin{subfigure}[b]{0.45\linewidth}
    \includegraphics[width=\linewidth]{Imagenes/4_MONACITA/WIREFRAME/MONACITA_c.png}
    \caption{Proyección del esqueleto}
  \end{subfigure}
  \caption{Proyección de la monacita a lo largo del eje \(c\), mostrando la disposición característica del grupo espacial \(P2_1/n\).}
  \label{fig:monacita_big_c}
\end{figure}

\begin{figure}[H]
  \centering
  \includegraphics[width=0.8\textwidth]{Imagenes/4_MONACITA/MONACITA.png}
  \caption{Vista tridimensional de la monacita mostrando la conectividad entre eneacedros \ce{CeO9} (amarillo) y tetraedros \ce{PO4} (naranja).}
  \label{fig:monacita_pol_perspective}
\end{figure}

\section{Rhabdofano (\texorpdfstring{\ce{LnPO4*nH2O}}{LnPO4*nH2O}, Ln= Y, Tb-Lu)}
El Rhabdofano, análogamente al par Xenótimo-Churchita, al calentarse (500–900 °C) colapsa irreversiblemente en la estructura Monacita \cite{chong_synthesis_2024,shelyug_thermodynamics_2018}. Sin embargo, a diferencia de la estructura laminar de la Churchita, el Rhabdofano adopta una arquitectura hexagonal (\(P6_222\)) con canales zeolíticos abiertos en los cuales se alojan moléculas de agua estructurales, lo que permite variabilidad en su hidratación \cite{enikeeva_structure_2023,rafiuddin_review_2022}. El Rhabdofano es un fosfato hidratado de tierras raras ligeras (Ln = La-Gd) de mayor radio iónico.

Los parámetros cristalográficos para la representación del Rhabdofano con \ce{CePO4*0.67H2O} se obtienen de ``Mindat.org'' \footnote{\url{https://www.mindat.org/min-3397.html}} \cite{Mooney_rhabdophane}.

La estructura del Rhabdofano se caracteriza por un sistema cristalino hexagonal con grupo espacial \({P 6_2 2 2}\) (No. 180), lo que indica una celda de tipo primitiva (\(P\)). \(6_2\) indica un eje helicoidal de \(60^\circ\) con una traslación de 2/6 de la altura de la celda unitaria, y los dos \(2\) indican ejes de rotación de \(180^\circ\) perpendiculares al eje principal \(c\). El grupo puntual \(6mm\): \(6\) indica un eje de rotación de \(60^\circ\) y las dos \(m\) indican planos de simetría verticales al eje principal \(c\), uno que pasa por los ejes \(a\) y \(b\) y otro que pasa por las diagonales del plano \(ab\).

Las coordenadas atómicas, factores de ocupación y parámetros térmicos isotrópicos para la estructura del Rhabdofano se presentan en la Tabla \ref{tab:coord_rhabdofano}.

\begin{table}[H]
  \centering
  \caption{Parámetros cristalográficos del rhabdofano (\ce{CePO4*0.67H2O}): coordenadas fraccionarias, ocupación, desplazamientos térmicos, posiciones de Wyckoff y simetría puntual.}
  \label{tab:coord_rhabdofano}
  \begin{tabular}{ l c c c c c c c}
    \toprule
    Átomo & \(x\) & \(y\) & \(z\) & Ocup. & \(U\) & Pos. Wyckoff & Sim.    \\
    \midrule
    Ce    & 0.500 & 0.000 & 0.000 & 1     & 0.000 & \(3c\)       & \(222\) \\
    P     & 0.500 & 0.000 & 0.500 & 1     & 0.000 & \(3d\)       & \(222\) \\
    O     & 0.446 & 0.147 & 0.360 & 1     & 0.000 & \(12k\)      & \(1\)   \\
    Ow    & 0.000 & 0.000 & 0.000 & 1     & 0.000 & \(3a\)       & \(222\) \\
    \bottomrule
  \end{tabular}
\end{table}

Los parámetros de la celda unitaria son \(a = b \neq c\) con \(a=7.0550\) \si{\angstrom} y los ángulos \(\alpha = \beta = 90^\circ\), \(\gamma = 120^\circ\). El volumen de la celda unitaria es \(277.551182\) \si{\cubic\angstrom}. Cada celda unitaria contiene 3 unidades de fórmula (\(Z=3\)); por lo tanto, la fórmula empírica es \ce{Ce3P3O12*2H2O}. La celda unitaria se muestra en las proyecciones a lo largo de los ejes \(a\), \(b\) y \(c\) en la Figura \ref{fig:rhabdofano_uc}.

\begin{figure}[H]
  \centering
  \begin{subfigure}[b]{0.3\linewidth}
    \includegraphics[width=\linewidth]{Imagenes/5_RHABDOFANO/ATOMS/RHABDOFANO_a.png}
    \caption{Eje \(a\)}
    \label{fig:rhabdofano_atoms_a}
  \end{subfigure}
  \hfill
  \begin{subfigure}[b]{0.3\linewidth}
    \includegraphics[width=\linewidth]{Imagenes/5_RHABDOFANO/ATOMS/RHABDOFANO_b.png}
    \caption{Eje \(b\)}
    \label{fig:rhabdofano_atoms_b}
  \end{subfigure}
  \hfill
  \begin{subfigure}[b]{0.3\linewidth}
    \includegraphics[width=\linewidth]{Imagenes/5_RHABDOFANO/ATOMS/RHABDOFANO_c.png}
    \caption{Eje \(c\)}
    \label{fig:rhabdofano_atoms_c}
  \end{subfigure}
  \caption{Proyecciones de la celdilla unidad del rhabdofano. Código de colores: Ce (amarillo), P (naranja), O (rojo), \ce{H2O} (azul).}
  \label{fig:rhabdofano_uc}
\end{figure}

\vspace{0.5cm}

\begin{minipage}[c]{0.70\textwidth} % Columna de TEXTO
  \setlength{\parskip}{1em}
  El entorno de coordinación del fósforo en el Rhabdofano destaca por su extraordinaria regularidad geométrica. A diferencia de la fase hidratada monoclínica (Churchita), este tetraedro \ce{PO4} es cristalográficamente perfecto en cuanto a sus longitudes de enlace, presentando cuatro distancias P--O idénticas de \SI{1.558}{\angstrom}. Esta simetría ideal se refleja en un índice de distorsión de longitud de enlace nulo (0.0000) y una elongación cuadrática de 1.0000. La varianza angular es prácticamente despreciable (\SI{0.018}{\degree\squared}), lo que confirma que el poliedro no sufre deformaciones angulares significativas. El volumen calculado para este tetraedro es de \SI{1.94}{\cubic\angstrom}, ligeramente mayor que el observado en las fases anhidras como el Xenótimo (\SI{1.84}{\cubic\angstrom}).
\end{minipage}
\hfill
\begin{minipage}[c]{0.25\textwidth}
  \begin{figure}[H]
    \centering
    \includegraphics[width=\linewidth]{Imagenes/5_RHABDOFANO/RHABDOFANO_Td.png}
    \captionsetup{justification=centering}
    \caption{Tetraedro \ce{PO4} del rhabdofano con cuatro enlaces P--O equivalentes de \SI{1.54}{\angstrom}, reflejando la alta simetría hexagonal \(P6_222\).}
    \label{fig:rhabdofano_td}
  \end{figure}
\end{minipage}%

\vspace{0.5cm}

\begin{minipage}[c]{0.3\textwidth} % Columna de IMAGEN (Izquierda)
  \centering
  \begin{figure}[H]
    \includegraphics[width=\linewidth]{Imagenes/5_RHABDOFANO/RHABDOFANO_Dode.png}
    \captionsetup{justification=centering}
    \caption{Antiprisma cuadrado \ce{CeO8} del rhabdofano con ocho distancias Ce--O equivalentes de \SI{2.53}{\angstrom}, característico de un poliedro altamente simétrico.}
    \label{fig:rhabdofano_Dode}
  \end{figure}
\end{minipage}%
\hfill
\begin{minipage}[c]{0.65\textwidth} % Texto a todo lo ancho
  \setlength{\parskip}{1em}
  Por otro lado, el catión cerio (\ce{Ce^{3+}}) se encuentra coordinado por ocho átomos de oxígeno pertenecientes a los grupos fosfato, conformando un poliedro \ce{CeO8} con geometría de antiprisma cuadrado distorsionado. A diferencia del entorno del fósforo, este sitio catiónico presenta una fuerte distorsión, evidenciada por un índice de 0.0631. El análisis de las distancias de enlace revela dos conjuntos claramente diferenciados: cuatro enlaces Ce--O más cortos de \SI{2.330}{\angstrom} y cuatro enlaces significativamente más largos de \SI{2.644}{\angstrom}. Esta disparidad resulta en una distancia media de enlace de \SI{2.487}{\angstrom} y un número de coordinación efectivo de 6.64, lo que sugiere que, aunque geométricamente hay 8 oxígenos vecinos, la interacción es mucho más fuerte con los cuatro más cercanos. El volumen del poliedro es de \SI{25.95}{\cubic\angstrom}, siendo el más voluminoso de las estructuras estudiadas, consistente con la naturaleza abierta de la estructura del Rhabdofano que aloja canales zeolíticos.
\end{minipage}

\vspace{0.5cm}

En esta estructura, los iones \ce{Ce^{3+}} están coordinados por ocho átomos de oxígeno y moléculas de agua, formando poliedros \ce{CeO8*0.67H2O} que se disponen alrededor de los tetraedros \ce{PO4}.

Las representaciones de los poliedros \ce{CeO8*0.67H2O} y \ce{PO4}, así como del esqueleto de la estructura, se muestran en las Figuras \ref{fig:rhabdofano_big_a}, \ref{fig:rhabdofano_big_b} y \ref{fig:rhabdofano_big_c}.

\begin{figure}[H]
  \centering
  \begin{subfigure}[b]{0.45\linewidth}
    \includegraphics[width=\linewidth]{Imagenes/5_RHABDOFANO/POLYHEDRA/RHABDOFANO_a.png}
    \caption{Proyección de los poliedros}
  \end{subfigure}
  \hfill
  \begin{subfigure}[b]{0.45\linewidth}
    \includegraphics[width=\linewidth]{Imagenes/5_RHABDOFANO/WIREFRAME/RHABDOFANO_a.png}
    \caption{Proyección del esqueleto}
  \end{subfigure}
  \caption{Representaciones de los poliedros y del esqueleto en la celdilla unitaria del Rhabdofano a lo largo del eje \(a\).}
  \label{fig:rhabdofano_big_a}
\end{figure}

\begin{figure}[H]
  \centering
  \begin{subfigure}[b]{0.45\linewidth}
    \includegraphics[width=\linewidth]{Imagenes/5_RHABDOFANO/POLYHEDRA/RHABDOFANO_b.png}
    \caption{Proyección a lo largo del eje \(b\).}
  \end{subfigure}
  \hfill
  \begin{subfigure}[b]{0.45\linewidth}
    \includegraphics[width=\linewidth]{Imagenes/5_RHABDOFANO/WIREFRAME/RHABDOFANO_b.png}
    \caption{Proyección a lo largo del eje \(b\).}
  \end{subfigure}
  \caption{Representaciones de los poliedros y esqueletos del Rhabdofano a lo largo del eje \(b\).}
  \label{fig:rhabdofano_big_b}
\end{figure}

\begin{figure}[H]
  \centering
  \begin{subfigure}[b]{0.45\linewidth}
    \includegraphics[width=\linewidth]{Imagenes/5_RHABDOFANO/POLYHEDRA/RHABDOFANO_c.png}
    \caption{Proyección de los poliedros}
  \end{subfigure}
  \hfill
  \begin{subfigure}[b]{0.45\linewidth}
    \includegraphics[width=\linewidth]{Imagenes/5_RHABDOFANO/WIREFRAME/RHABDOFANO_c.png}
    \caption{Proyección del esqueleto}
  \end{subfigure}
  \caption{Representaciones de los poliedros y del esqueleto en la celdilla unitaria del Rhabdofano a lo largo del eje \(c\).}
  \label{fig:rhabdofano_big_c}
\end{figure}


\begin{figure}[H]
  \centering
  \includegraphics[width=0.8\textwidth]{Imagenes/5_RHABDOFANO/RHABDOFANO.png}
  \caption{Representación de los poliedros en perspectiva del Rhabdofano.}
  \label{fig:rhabdofano_pol_perspective}
\end{figure}

\chapter{Conclusiones}
El análisis cristalográfico comparativo de los ortofosfatos de lantánidos (\ce{LnPO4}) revela la influencia determinante del radio iónico y la hidratación sobre la estructura cristalina adoptada. El Zircón (\ce{ZrSiO4}) actúa como aristotipo para el Xenótimo, compartiendo el grupo espacial tetragonal \(I4_1/amd\) y número de coordinación 8. La contracción lantánida permite que los lantánidos pesados (HREE: Y, Tb-Lu) adopten esta estructura compacta de alta simetría.

Por el contrario, los lantánidos ligeros (LREE: La-Gd), de mayor radio iónico, requieren la estructura monoclínica Monacita (\(P2_1/n\)) con número de coordinación 9, evidenciando una transición morfotrópica impulsada por factores estéricos. La hidratación introduce una complejidad adicional: la Churchita (monoclínica \(C2/c\)) y el Rhabdofano (hexagonal \(P6_222\)) son fases metaestables precursoras del Xenótimo y Monacita, respectivamente. Ambas incorporan moléculas de agua en su red cristalina, modificando la simetría y el entorno de coordinación.

El estudio de estas modificaciones de simetría es fundamental para comprender las propiedades tecnológicas de estos materiales, incluyendo su estabilidad térmica, resistencia a la radiación y eficiencia fotoluminiscente, lo que justifica su aplicación en gestión de residuos nucleares, cerámicas avanzadas y dispositivos optoelectrónicos.

\clearpage
\addcontentsline{toc}{chapter}{Bibliografía} % Se añade al índice como un elemento no numerado
\printbibliography
\nocite{*}

\end{document}
