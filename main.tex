% !TeX program = lualatex
\documentclass[12pt, a4paper]{report}

% *1. Configuración de Idioma y Fuentes (Guía Docente)

 \usepackage{fontspec}
 \usepackage{microtype}
 \usepackage{polyglossia}
 \usepackage[version=4]{mhchem}
 \setmainlanguage{spanish}
 \setotherlanguage{english}

  % ?  REQUISITO FCQ: Fuente Times New Roman (Debe estar instalada en el sistema)
   \setmainfont{TIMES.TTF}[
    Path = ./fonts/,
    BoldFont = TIMESBD.TTF,
    ItalicFont = TIMESI.TTF,
    BoldItalicFont = TIMESBI.TTF]


%* 2. Formato de Página (Guía Docente)

  \usepackage[left=2.5cm, right=2.5cm, top=2.5cm, bottom=2.5cm]{geometry} % MÁRGENES 2.5 CM (REQUISITO FCQ)
  \usepackage{setspace}
  \onehalfspacing % INTERLINEADO 1.5 LÍNEAS (REQUISITO FCQ)

  %? PÁRRAFOS: Sin sangría, con espacio vertical entre ellos
   \setlength{\parindent}{0pt} % ELIMINA LA SANGRÍA (REQUISITO DEL USUARIO)
   \setlength{\parskip}{1em}  % Espacio vertical entre párrafos

  %? Para incluir PDFs (Portada y Declaración IA)
    \usepackage{pdfpages} 
    \usepackage{afterpage}
    \usepackage[hidelinks]{hyperref}

%* 3. Configuración de Capítulos (titlesec)
  \usepackage{caption} 
  \captionsetup[table]{name=Tabla}  
  \usepackage{titlesec}

  %? REQUISITO USUARIO: Capítulo sin la palabra "Capítulo", sin espacios grandes
    \titleformat{\chapter}[hang]{\normalfont\bfseries\Huge}{\thechapter.}{0.5em}{\bfseries\Huge} 
    \titlespacing*{\chapter} {0pt}{0pt}{1.5\baselineskip} % Sin espacio antes (0pt), sin espacio después (0pt), pero dejamos un espacio de una línea y media antes del texto para legibilidad.

%* 4. Contenido Académico y Citas

  \usepackage{amsmath, amssymb}
  \usepackage{siunitx} % Unidades
  \DeclareSIUnit{\angstrom}{\text{\AA}} % Definición del símbolo Angstrom
  \usepackage{graphicx}
  \usepackage{wrapfig}
  \usepackage{subcaption} 
  \usepackage{longtable}

%* 4. Configuración de Citas y Bibliografía (APA Numerado)

  \usepackage[
    backend=biber,
    style=apa,        % Usamos el estilo APA para la bibliografía
    sorting=nyt,      % Ordenar por Nombre, Año, Título
    language=spanish,
    citestyle=numeric % Forzamos que las citas en el texto sean números [1]
  ]{biblatex}

    %* 4.1. Quitar la "sangría francesa" típica de APA (margen izquierdo a 0)
 \setlength{\bibhang}{0pt}

    %* 4.2. Añadir espacio entre cada referencia
    \setlength{\bibitemsep}{1.5\baselineskip}
    \ExecuteBibliographyOptions{labelnumber}
    \DeclareFieldFormat{labelnumberwidth}{\mkbibbrackets{#1}}
    \defbibenvironment{bibliography}
    {\list
        {\printtext[labelnumberwidth]{%
        \printfield{labelnumber}}}
        {\setlength{\labelwidth}{\labelnumberwidth}%
      \setlength{\leftmargin}{\labelwidth}%
      \setlength{\labelsep}{\biblabelsep}%
      \addtolength{\leftmargin}{\labelsep}%
      \setlength{\itemsep}{1.5\baselineskip} 
      \setlength{\parsep}{\bibparsep}}}
    {\endlist}
    {\item}

 \addbibresource{Bibliografia.bib}

%* 5. PERSONALIZACIÓN DE FIGURAS (REQUISITO TUTORA: Figura 1, 2, 3...)

  \usepackage{chngcntr}
  \counterwithout{figure}{chapter}
  \counterwithout{table}{chapter}
  \usepackage{enumitem}
  \usepackage{multirow}
  \usepackage{booktabs}
  \usepackage{tabularx}
  \usepackage{float}

  % ? Solución para que los enlaces NO se salgan del margen ---
    \setcounter{biburllcpenalty}{7000}
    \setcounter{biburlucpenalty}{8000}
    \setcounter{biburlnumpenalty}{9000}

%? META-DATOS DEL DOCUMENTO
  \hypersetup{
    pdftitle={ESTUDIO DE ALGUNAS ESTRUCTURAS DE COMPUESTOS INORGÁNICOS Y MODIFICACIONES DE SIMETRÍA}, % Pon el título exacto de tu TFG
    pdfauthor={Lucia Lan Burillo Blazquez},       % Tu nombre completo
    pdfsubject={Trabajo de Fin de Grado - Grado en Química}, % La asignatura o el tipo de trabajo
    pdfkeywords={Cristalografía, Lantánidos, VESTA, Zircón, Monacita, Xenótimo, Rhabdofano, Simetría}, % Palabras clave importantes
    pdfcreator={LaTeX},                 % Opcional: indica con qué se creó
    pdfproducer={luLaTeX},             % Opcional
    pdfdisplaydoctitle=true             % Muestra el título del documento en la barra de ventana en vez del nombre del archivo
  }

%* 6. INICIO DEL DOCUMENTO

\begin{document}


% !A. PORTADA (Num. no visible)

%! REQUISITO FCQ: Insertar la portada como PDF (DOCX convertido a PDF)
%!\includepdf[pages=1]{gq-portada-tfg-2024-25.pdf} 

\clearpage

%! DECLARACIÓN RESPONSABLE DE AUTORÍA (OBLIGATORIO)
%! REQUISITO FCQ: Insertar el documento oficial (firmado) como PDF
%!\includepdf[pages=1, pagecommand={\thispagestyle{plain}}]{declaracion-responsable-sobre-autoria-y-uso-etico-de-herramientras-de-inteligencia-artificial-06062025.pdf} 


% ÍNDICES (COMIENZAN LA NUMERACIÓN )
\pagenumbering{arabic}
\setcounter{page}{1}
\tableofcontents 
\clearpage

% C. CUERPO PRINCIPAL DEL TFG

\chapter{Introducción}
\setmainlanguage{english}

\captionsetup[table]{name=Table}   % Fuerza que se llame "Table" en esta sección
\captionsetup[figure]{name=Figure} % Fuerza que se llame "Figure" en esta sección

\section{Theoretical Background}
 A crystal is defined as a solid material whose atoms are arranged in an orderly repeating pattern extending in all three spatial dimensions \cite{ashcroft1976solid}. The study of crystal structures is fundamental in materials science, as the arrangement of atoms directly influences the physical and chemical properties of materials \cite{callister2020materials}.

 The periodic order of the atoms is evident in different directions around each atom. The directions are non-coplanar and originate at each atom of the crystal. The axes are chosen by symmetry criteria. Once the axes are defined, the unit cell is established as the smallest portion of the crystal that, when repeated in space through translations along the axes, can recreate the entire crystal structure. Its geometry is defined by reticular parameters: the lengths of the edges (\(a\), \(b\), \(c\)) and the angles between them (\(\alpha\), \(\beta\), \(\gamma\)).
 
 There are seven crystal systems based on the relationships between the unit cell parameters: cubic, tetragonal, orthorhombic, hexagonal, trigonal, monoclinic, and triclinic. Each system is characterized by specific constraints on the lengths and angles of the unit cell.

 \begin{table}[H]
    \centering
    \caption{Crystal systems and their unit cell parameters.}
    \label{tab:crystal_systems}
    \vspace{0.1cm}
    \begin{tabular}{l c c}
        \toprule
        System & Dimensions & Angles \\
        \midrule
        Cubic        & \(a = b = c\)       & \(\alpha = \beta = \gamma = 90^\circ\) \\
        Tetragonal   & \(a = b \neq c\)    & \(\alpha = \beta = \gamma = 90^\circ\) \\
        Orthorhombic & \(a \neq b \neq c\) & \(\alpha = \beta = \gamma = 90^\circ\) \\
        Hexagonal    & \(a = b \neq c\)    & \(\alpha = \beta = 90^\circ\), \(\gamma = 120^\circ\) \\
        Rhombohedral & \(a = b = c\)       & \(\alpha = \beta = \gamma \neq 90^\circ\) \\
        Monoclinic   & \(a \neq b \neq c\) & \(\alpha = \gamma = 90^\circ\), \(\beta \neq 90^\circ\) \\
        Triclinic    & \(a \neq b \neq c\) & \(\alpha \neq \beta \neq \gamma \neq 90^\circ\) \\
        \bottomrule
    \end{tabular}
 \end{table}

 The arrangement of atoms within the unit cell defines 14 distinct Bravais lattices, categorized into seven crystal systems. These lattices are distinguished by their centering: primitive (P), body-centered (I), face-centered (F), and base-centered (C).

 In primitive (P) lattices, atoms are located exclusively at the corners of the unit cell. Body-centered (I) lattices feature atoms at the corners plus an additional atom at the geometric center of the cell. Face-centered (F) lattices have atoms positioned at the corners and at the center of each face, whereas base-centered (C) lattices contain atoms at the corners and at the centers of two opposite faces. 

 The 14 Bravais lattices are illustrated in Figure \ref{fig:Bravais_lattices}.

 \begin{figure}[H]
    \centering
    \includegraphics[width=0.7\textwidth]{Imagenes/Bravais_Lattices.jpeg}
      \caption{Bravais Lattices representing the 14 distinct lattice types in three-dimensional space.} 
    \label{fig:Bravais_lattices}
 \end{figure}
 
 Lattice planes are defined by Miller indices (hkl), which are a set of three integers that denote the orientation of the planes in the crystal lattice. These indices are derived from the reciprocals of the fractional intercepts that the plane makes with the crystallographic axes.

 Symmetry operations are transformations that map a crystal structure onto itself, preserving its overall arrangement. These operations include rotations, reflections, inversions, and translations. The combination of these symmetry operations defines the space group of a crystal, which provides a comprehensive description of its symmetry properties.

 \begin{itemize} 
   \item Rotations:
     \begin{itemize}
      \item Rotational axes \((n)\) are the ones around which the crystal can be rotated by specific angles and still appear unchanged. Common rotation axes include 2-fold (180°), 3-fold (120°), 4-fold (90°), and 6-fold (60°) rotations.
      \item Rotation-reflection combined symmetry operations involve a rotation followed by a reflection across a plane. These operations can create complex symmetry elements in crystals.
     \end{itemize}

   \item Reflections:
     \begin{itemize} 
      \item Mirror planes \((m)\) are planes that divide the crystal into two symmetrical halves, where one half is the mirror image of the other.
     \end{itemize}

   \item Inversions: 
     \begin{itemize}
      \item Inversion centers (\(\bar{1}\) or \(i\)). Inversion through a central point maps every point \((x, y, z)\) to \((-x, -y, -z)\).
     \item Rotoinversion axes \((\bar{n})\) involve a rotation followed by an inversion through a point. Common rotoinversion axes include \(\bar{3}\) (rotation + inversion) and \(\bar{4}\).
     \end{itemize}
   \item Translations: represent a shift of the entire crystal lattice by a specific vector.
     \begin{itemize}
      \item Helicoidal axes (Screw axes) represent a combined rotation and translation along the axes. \(X_n\) where \(n < X\). Types of screw axes include:
       \begin{itemize}
        \item 2\(_1\): 180° rotation + translation of 1/2 along the axes.
        \item 3\(_1\): 120° rotation + translation of 1/3 along the axes.
        \item 3\(_2\): 240° rotation + translation of 2/3 along the axes.
       \end{itemize}
      \item Glide planes combine reflection with translation parallel to the plane. Common types include \(a\)-, \(b\)-,and \(c\)- glides (translation \(1/2\) along the corresponding axis), the \(n-\) glide (\(1/2\) along face diagonal), and \(d\)-glide (diamond glide, (\(1/4\)) translation along the face diagonal).
     \end{itemize} 
 \end{itemize}
 
 Platonic solids are highly symmetrical, three-dimensional shapes with identical faces, edges, and angles. In crystallography, the five Platonic solids (tetrahedron, cube, octahedron, dodecahedron, and icosahedron) can be used to describe the coordination environments of atoms within a crystal structure. 

 Point groups classify the symmetry of objects based on their rotational and reflectional symmetries, excluding translations. There are 32 distinct point groups in three-dimensional space, each characterized by a unique combination of symmetry elements. Hermann-Mauguin notation is a symbolic system used to describe the symmetry elements of crystals, including point groups and space groups. This notation provides a concise way to represent the symmetry operations present in a crystal structure.

 The 230 space groups classify the symmetry of crystal structures, combining translational and point symmetries. Each space group is identified by a unique number and symbol, providing a comprehensive description of the symmetry operations that can be applied to the crystal lattice.

 International crystallographic tables indicate for each atom the following information:
 \begin{itemize} [label=\textendash]
  \item Coordinates indicate the position within the unit cell, expressed as fractions of the cell parameters (\(a\), \(b\), \(c\)). 
  \item Occupancy factors represent the proportion of a specific atomic site that is occupied, ranging from 0 (completely unoccupied) to 1 (fully occupied). 
  \item Isotropic thermal parameters (\(U\)) describe the average displacement of atoms from their mean positions due to thermal vibrations, assuming uniform movement in all directions.
  \item Wyckoff positions describe the specific locations of atoms within a unit cell based on the symmetry of the space group. Each Wyckoff position is associated with a multiplicity (the number of equivalent positions generated by symmetry operations) and a letter (such as \(a\), \(b\), \(c\), etc.) that indicate the symmetry of the site, \(a\) corresponding to the highest symmetry site in the unit cell. For example, \(4a\) indicates four equivalent positions with the highest symmetry.
  \item Symmetry indicates whether there is or not (\(1\)) a symmetry element with its symbol.
 \end{itemize}

 \section{State of Art}
   The study of the crystal structures presented above is particularly relevant when analyzing lanthanide orthophosphates (\ce{LnPO_4}), a class of materials with rich structural chemistry and diverse technological applications. These compounds are fundamental in fields such as photonics, catalysis, radioactive waste management, and biomedical imaging \cite{rafiuddin_review_2022}. The selection of the specific crystalline structures—whether the anhydrous form (Monazite, Xenotime) or the hydrated form (Rhabdophane, Churchite)—is critical, as the crystal structure dictates key properties such as luminescence intensity, chemical stability, and magnetic behavior \cite{garrido_hernandez_photoluminescence_2016, rafiuddin_structural_2022}.

   The inclusion of the Zircon (\ce{ZrSiO_4}) structure in this study is fundamental for two crystallographic and comparative reasons:

    \begin{enumerate}
     \item Structural Prototype: Zircon serves as the crystallographic aristotype for the Xenotime phase. Both compounds are isostructural, crystallizing in the same tetragonal system with the \(I4_1/amd\) space group \cite{chong_synthesis_2024} . Understanding the arrangement of \ce{ZrO8} and \ce{SiO_4} polyhedra in Zircon is a prerequisite for analyzing the geometry of heavy lanthanide phosphates (\ce{HREEPO_4}), where \ce{Ln^{3+}} and \ce{P^{5+}} occupy the positions of \ce{Zr^{4+}} and \ce{Si^{4+}}, respectively.
     \item Stability Benchmark: Zircon is widely recognized as a benchmark material for evaluating radiation resistance in nuclear wasteforms \cite{rafiuddin_review_2022}. Comparative studies have demonstrated that under irradiation, silicate minerals with the Zircon structure undergo amorphization and lack structural recovery mechanisms \cite{chong_synthesis_2024}. In contrast, the analogous phosphate structures (Xenotime and Monazite) exhibit ``self-healing'' capabilities and superior resistance to amorphization at lower temperatures, justifying the specific interest in phosphates over silicates for the long-term storage of actinides \cite{chong_synthesis_2024}.
   \end{enumerate}

   Polymorphism and Stability Domains The crystallization of these phosphates into a specific structure depends primarily on the ionic radius of the lanthanide cation (\ce{Ln^{3+}}) and synthesis conditions, such as temperature and pH \cite{enikeeva_structure_2023}. There is a distinct structural dependence based on the contraction of the lanthanide series:

   \begin{itemize}[label=\textendash]
     \item Light Lanthanides (\ce{LREE}), e.g., La-Gd: Tend to crystallize in the monoclinic Monazite type structure or its hexagonal hydrated form, Rhabdophane (\ce{LnPO_4*nH_2O}) \cite{chong_synthesis_2024}.
     \item Heavy Lanthanides (\ce{HREE}, e.g., Tb-Lu): Mainly crystallize in the tetragonal Xenotime phase or the monoclinic hydrated Churchite phase (\ce{LnP_4\cdot 2H_2O}) \cite{chong_synthesis_2024}.
     \item Transition Zone: Intermediate elements such as Gadolinium (\ce{Gd}), Terbium (\ce{Tb}), or Dysprosium (\ce{Dy}) can adopt multiple structures depending on conditions. A ``critical radius'' exists for the transition between monazite and xenotime structures, typically located between the ionic radii of \ce{Tb} and \ce{Gd} \cite{mogilevsky_miscibility_2007}.
     \item Recent thermodynamic studies have indicated that rhabdophane phases are generally metastable with respect to the corresponding monazite plus water at all temperatures under ambient pressure; however, rhabdophane often precipitates first due to kinetic controls \cite{shelyug_thermodynamics_2018}.
     \item Phase Transformations and Dehydration There is a direct transformation relationship between low-temperature hydrated phases and high-temperature stable anhydrous phases. For instance, the Rhabdophane structure is metastable and, upon heating, loses its structural water (zeolitic water in the channels) to irreversibly transform into the Monazite structure. This transformation typically occurs between 500 and 600 °C, although the dehydration process can begin at lower temperatures \cite{kenges_synthesis_2017,shelyug_thermodynamics_2018}. Similarly, the Churchite phase is stable at low temperatures (typically synthesized via precipitation), but upon heating above 300 °C, it dehydrates and transforms directly into the Xenotime structure \cite{rafiuddin_structural_2022}. Controlling these transformations is vital, as the presence of water and the symmetry of the lanthanide ion's environment drastically affect properties such as photoluminescent efficiency; for example, the removal of \ce{-OH} defects during the transition to the tetragonal phase significantly improves emission intensity in \ce{Tb}-doped systems\cite{garrido_hernandez_photoluminescence_2016}.
   \end{itemize}

   Technological Interest and Applications Each of these structures offers specific advantages for industrial and scientific applications:
  

   Nuclear Waste Management: Monazite and Xenotime are considered excellent matrices for the long-term confinement of high-level nuclear waste (actinides) due to their high radiation resistance, thermal stability, and low solubility \cite{chong_synthesis_2024,rafiuddin_review_2022}. Monazite ceramics have demonstrated the ability to incorporate significant amounts of tetravalent and trivalent actinides, mimicking natural minerals that have retained radionuclides over geological timescales \cite{shelyug_thermodynamics_2018}.

     Structural Ceramics and Coatings: Due to their high melting points (>2000 °C), chemical stability, and compatibility with other oxides, anhydrous phases are investigated as fiber coatings in Ceramic Matrix Composites (CMCs) to improve fracture toughness and prevent oxidation \cite{chong_synthesis_2024}.

   Nanotechnology and Catalysis: The Rhabdophane structure is of significant interest due to its morphology. It has been reported that rhabdophane nanoparticles synthesized by hydrothermal methods can form single-crystal rods with internal mesoporosity or cavities, which is attractive for catalytic and adsorption applications \cite{enikeeva_structure_2023}.
   
   Biomedical Imaging: Recent investigations highlight the potential use of Churchite materials as contrast agents in Magnetic Resonance Imaging (MRI). Magnetic susceptibility measurements have shown that these hydrated phases exhibit effective magnetic moments similar to free lanthanide ions, making them suitable candidates for such applications \cite{rafiuddin_review_2022}.

\chapter{Objetivo}
  El objetivo principal de este trabajo es el estudio detallado de las estructuras cristalinas de varios compuestos inorgánicos, específicamente el Zircón (\ce{ZrSiO4}), el Xenótimo (\ce{LnPO4}, Ln = Y, Tb-Lu), la Monacita (\ce{LnPO4}, Ln = La-Gd), el Rhabdofano (\ce{LnPO4*nH2O}, Ln = La-Gd) y la Churchita (\ce{LnPO4*2H2O}, Ln = Tb-Lu). Analizando estas estructuras utilizando el software VESTA para la visualización y análisis cristalográfico. Además, se explorarán las modificaciones de simetría en estas estructuras y su relación con las propiedades físicas y químicas de los materiales estudiados.

\chapter{Resultados y Discusión}
 \captionsetup[table]{name=Tabla}   
 \captionsetup[figure]{name=Figura}

 Empleando el software de visualización y análisis cristalográfico VESTA \cite{prog:vesta}, se generaron las representaciones gráficas de las estructuras cristalinas de los compuestos estudiados. A continuación, se detallan las características estructurales de cada uno de los materiales analizados, incluyendo sus parámetros de celda, coordenadas atómicas, factores de ocupación y parámetros térmicos isotrópicos.

 \section{Zircón (\texorpdfstring{\ce{ZrSiO4}}{ZrSiO4})}
    
   Los parámetros cristalográficos del Zircón \ce{ZrSiO4} se obtienen de ``Materials Data on \ce{ZrSiO4} by Materials Project''\footnote{\url{https://next-gen.materialsproject.org/materials/mp-4820}} \cite{zircon_data_2020}.
   
   La estructura del Zircón, se caracteriza por un sistema cristalino tetragonal con grupo espacial \(I4_1/amd\) (No. 141),  I indica un átomo centrado en la celda, 4\(_1\) indica un eje de \(90^\circ\) de rotación con una traslación de 1/4 de la altura de la celda unitaria, \(a\) indica un plano de simetría perpendicular al eje principal, \(m\) indica un plano de simetría y \(d\) indica un plano de simetría diagonal. El grupo puntual \(4/mmm\), \(4/m\) indica un eje de rotación de \(90^\circ\) con un plano de simetría horizontal perpendicular al eje principal \(c\). La segunda \(m\) indica  un plano de simetría vertical al eje principal pasando por los ejes \(a\) y \(b\). La tercera \(m\) indica un plano de simetría vertical al eje principal pasando por las diagonales del plano \(ab\). Con \(Z=4\) unidades de fórmula por celda unitaria.
  
   Las coordenadas atómicas, factores de ocupación y parámetros térmicos isotrópicos para la estructura del Zircón se presentan en la Tabla \ref{tab:coord_Zircón}.

  \begin{table}[H]
   \centering
   \caption{Coordenadas atómicas, factores de ocupación y parámetros térmicos isotrópicos para la estructura del Zircón.}
   \label{tab:coord_Zircón}
   \vspace{0.1cm}
   \begin{tabular}{l c c c c c c c}
   \toprule
    Átomo & \(x\) & \(y\) & \(z\) & Ocup & \(U\) & Pos. Wyckoff & Sim. \\
   \midrule
   Zr & 0.0000 & 0.7500 & 0.1250 & 1 & 0.002 & \(4a\)  & \(\bar{4}m2\) \\
   Si & 0.0000 & 0.7500 & 0.6250 & 1 & 0.004 & \(4b\)  & \(\bar{4}m2\) \\
   O  & 0.0000 & 0.0661 & 0.1953 & 1 & 0.007 & \(16h\) & \(.m.\) \\
   \bottomrule
    \end{tabular}
 \end{table}

 Los parámetros de la celda unitaria son \(a = b \neq c\)  con \(a=b=6.62, c = 6.00\) \si{\angstrom} y los ángulos \(\alpha = \beta = \gamma = 90^\circ\).  El volumen de la celda unitaria es \(286.533\) \si{\cubic\angstrom}. Cada celda unitaria contiene 4 unidades de fórmula (\(Z=4\)), por lo tanto la fórmula empírica es \ce{Zr4Si4O16}. La celda unitaria se muestra en las proyecciones a lo largo de los ejes \(a\), \(b\) y \(c\) en la Figura \ref{fig:zircon_uc}.

 \begin{figure}[H]
   \centering
   \begin{subfigure}[b]{0.3\linewidth}
   \includegraphics[width=\linewidth]{Imagenes/1_ZIRCON/ATOMS/ZIRCON_a.png}
   \caption{Eje \(a\)}
   \label{fig:zircon_atoms_a}
   \end{subfigure}
     \hfill
   \begin{subfigure}[b]{0.3\linewidth}
   \includegraphics[width=\linewidth]{Imagenes/1_ZIRCON/ATOMS/ZIRCON_b.png}
   \caption{Eje \(b\)}
   \label{fig:zircon_atoms_b}
   \end{subfigure}
   \hfill
   \begin{subfigure}[b]{0.3\linewidth}
   \includegraphics[width=\linewidth]{Imagenes/1_ZIRCON/ATOMS/ZIRCON_c.png}
   \caption{Eje \(c\)}
   \label{fig:zircon_atoms_c}
   \end{subfigure}
   \caption{Proyecciones de la celdilla unidad}
   \label{fig:zircon_uc}
  \end{figure}

 \vspace{0.5cm} 
 
 \begin{minipage}[c]{0.70\textwidth} % Columna de TEXTO (Izquierda)
   \setlength{\parskip}{1em} % Mantiene espacio entre párrafos
   El entorno de coordinación del silicio conforma un tetraedro \ce{SiO4} aislado y cristalográficamente perfecto en cuanto a sus distancias de enlace, presentando cuatro enlaces Si--O idénticos de \SI{1.626}{\angstrom}. Esta uniformidad se refleja en un índice de distorsión de la longitud de enlace nulo (0.0000).

   Sin embargo, el poliedro no es un tetraedro geométrico ideal. El análisis de los ángulos de enlace revela una varianza angular de (\SI{97.71}{\degree})\(^2\) y un alargamiento cuadrático de 1.0238, valores que indican una compresión significativa del poliedro a lo largo del eje \(c\) (ver Figuras \ref{fig:Zircón_big_a} y \ref{fig:zircon_big_b}). El volumen calculado para este poliedro es de \SI{2.13}{\cubic\angstrom}.
 \end{minipage}% <--- ¡IMPORTANTE: No borres este símbolo de porcentaje!
   \hfill
 \begin{minipage}[c]{0.25\textwidth} % Columna de IMAGEN (Derecha)
    \centering
    \begin{figure}[H]
      \includegraphics[width=\linewidth]{Imagenes/1_ZIRCON/ZIRCON_Td.png}
      \captionsetup{justification=centering}
      \caption{Tetraedro \ce{SiO4}} 
    \label{fig:zircon_td}
    \end{figure}
 \end{minipage}

 \vspace{0.5cm} 

 \begin{minipage}[c]{0.3\textwidth}

    \centering
    \begin{figure}[H]
     \includegraphics[width=\linewidth]{Imagenes/1_ZIRCON/ZIRCON_Dode.png}
     \caption{Dodecaedro \ce{ZrO8}} 
     \label{fig:zircon_dode_fixed}
    \end{figure}
  \end{minipage}
    \hfill
 \begin{minipage}[c]{0.65\textwidth} % Columna de IMAGEN (Derecha)
   \setlength{\parskip}{1em} % Mantiene espacio entre párrafos
   Por su parte, el catión \ce{Zr^4+} presenta un número de coordinación 8, conformando un dodecaedro \ce{ZrO8} con una geometría más distorsionada que el tetraedro de silicio. A diferencia de este último, el entorno del zirconio no es equidistante: exhibe dos conjuntos de longitudes de enlace diferenciados, con cuatro enlaces más cortos de \SI{2.135}{\angstrom} y otros cuatro más largos de \SI{2.275}{\angstrom}.

   Esta disparidad se refleja en un índice de distorsión de la longitud de enlace de 0.0318 (frente al valor nulo observado para el Si). La distancia media de enlace Zr--O resultante es de \SI{2.205}{\angstrom}, y el poliedro ocupa un volumen considerablemente mayor, de \SI{19.18}{\cubic\angstrom}, consistente con el mayor radio iónico del zirconio (ver Figuras \ref{fig:Zircón_big_a} y \ref{fig:Zircón_big_c}).
 \end{minipage}

    \vspace{0.5cm} 
 
 En esta estructura, los iones \ce{Zr^4+} están coordinados por ocho átomos de oxígeno, formando poliedros \ce{ZrO8} que se disponen alrededor de los tetraedros \ce{SiO4}. Los tetraedros \ce{SiO4} están orientados de manera que sus vértices apuntan hacia los centros de los poliedros \ce{ZrO8}, creando una red tridimensional interconectada. Esta disposición resulta en una estructura robusta y estable, característica del Zircón. Las representaciones de los poliedros \ce{ZrO8} y \ce{SiO4}, así como del esqueleto de la estructura, se muestran en las Figuras \ref{fig:Zircón_big_a}, \ref{fig:zircon_big_b} y \ref{fig:Zircón_big_c}.

 \begin{figure}[H]
 \centering
    \begin{subfigure}[b]{0.45\linewidth}
    \includegraphics[width=\linewidth]{Imagenes/1_ZIRCON/POLYHEDRA/ZIRCON_a.png}
    \caption{Proyección de los poliedros}
    \end{subfigure}
      \hfill
    \begin{subfigure}[b]{0.45\linewidth}
    \includegraphics[width=\linewidth]{Imagenes/1_ZIRCON/WIREFRAME/ZIRCON_a.png}
    \caption{Proyección del esqueleto}
    \end{subfigure}
 \caption{Representaciones de los poliedros y del esqueleto en la celdilla unitaria del Zircón a lo largo del eje \(a\).}
 \label{fig:Zircón_big_a}
 \end{figure}

 \begin{figure}[H]
 \centering
    \begin{subfigure}[b]{0.45\linewidth}
    \includegraphics[width=\linewidth]{Imagenes/1_ZIRCON/POLYHEDRA/ZIRCON_b.png}
    \caption{Proyección a lo largo del eje \(b\).}
    \end{subfigure}
      \hfill
    \begin{subfigure}[b]{0.45\linewidth}
    \includegraphics[width=\linewidth]{Imagenes/1_ZIRCON/WIREFRAME/ZIRCON_b.png}
    \caption{Proyección a lo largo del eje \(b\).}
    \end{subfigure}
 \caption{Representaciones apliadas de los poliedros y esqueletos del Zircón a lo largo del eje \(b\).}
 \label{fig:zircon_big_b}
 \end{figure}
 

 \begin{figure}[H]
 \centering
    \begin{subfigure}[b]{0.45\linewidth}
    \includegraphics[width=\linewidth]{Imagenes/1_ZIRCON/POLYHEDRA/ZIRCON_c.png}
    \caption{Proyección de los poliedros}
    \end{subfigure}
      \hfill
    \begin{subfigure}[b]{0.45\linewidth}
    \includegraphics[width=\linewidth]{Imagenes/1_ZIRCON/WIREFRAME/ZIRCON_c.png}
    \caption{Proyección del esqueleto}
    \end{subfigure}
 \caption{Representaciones de los poliedros y del esqueleto en la celdilla unitaria del Zircón a lo largo del eje \(c\).}
 \label{fig:Zircón_big_c}
 \end{figure}
 

  \begin{figure}[H]
    \centering
    \includegraphics[width=0.8\textwidth]{Imagenes/1_ZIRCON/ZIRCON.png}
      \caption{Representación de los poliedros en perspectiva del Zircón.} 
    \label{fig:Zircón_pol_perspective}
 \end{figure}


 \section{Xenótimo (\texorpdfstring{\ce{LnPO4}}{LnPO4}, Ln = Y, Tb-Lu)}
   Los parámetros cristalográficos del Xenótimo \ce{YPO4} se obtienen de ``Mindat.org'' \footnote{\url{https://www.mindat.org/min-4333.html}} \cite{ni1995crystal}.

   Los fosfatos de tierras raras pesados (Tb-Lu, Y, Sc) cristalizan en la estructura del Xenótimo, isoesctructural con el Zircón \cite{chong_synthesis_2024}.

   La estructura del Xenótimo, isoesctructural con el Zircón,  se caracteriza por un sistema cristalino tetragonal con grupo espacial \(I4_1/amd\) (No. 141),  I indica un átomo centrado en la celda, \(4_1\) indica un eje de \(90^\circ\) de rotación con una traslación de 1/4 de la altura de la celda unitaria, \(a\) indica un plano de simetría perpendicular al eje principal, \(m\) indica un plano de simetría y \(d\) indica un plano de simetría diagonal. El grupo puntual \(4/mmm\), \(4/m\) indica un eje de rotación de \(90^\circ\) con un plano de simetría horizontal perpendicular al eje principal \(c\). La segunda \(m\) indica  un plano de simetría vertical al eje principal pasando por los ejes \(a\) y \(b\). La tercera \(m\) indica un plano de simetría vertical al eje principal pasando por las diagonales del plano \(ab\). Con \(Z=4\) unidades de fórmula por celda unitaria.

   Las coordenadas atómicas, factores de ocupación y parámetros térmicos isotrópicos para la estructura del Xenótimo se presentan en la Tabla \ref{tab:coord_xenotimo}.

   \begin{table}[H]
     \centering
     \caption{Coordenadas atómicas, factores de ocupación y parámetros térmicos isotrópicos para la estructura del Xenótimo.}
     \label{tab:coord_xenotimo}
     \vspace{0.1cm}
     \begin{tabular}{l c c c c c c c}
     \toprule
     Átomo & \(x\) & \(y\) & \(z\) & Ocup & \(U\) & Pos. Wyckoff & Sim. \\
     \midrule
      Y & 0.0000 & 0.7500 & 0.1250 & 1 & 0.004 & \(4a\) & \(\bar{4}m2\) \\
      P & 0.0000 & 0.2500 & 0.3750 & 1 & 0.005 & \(4b\) & \(\bar{4}m2\) \\
      O & 0.0000 & 0.0764 & 0.2175 & 1 & 0.008 & \(16h\) & \(.m.\) \\
     \bottomrule
     \end{tabular}
    \end{table}
 
 Los parámetros de la celda unitaria son \(a = b \neq c\)  con \(a=b=6.8947, c = 6.0276\) \si{\angstrom} y los ángulos \(\alpha = \beta = \gamma = 90^\circ\). La presencia de átomos de Y en vez de Zr provoca un aumento en el volumen de la celda unitaria respecto al Zircón, que en este caso es \(286.533\) \si{\cubic\angstrom}. Cada celda unitaria contiene 4 unidades de fórmula (\(Z=4\)), por lo tanto la fórmula empírica es \ce{Y4P4O16}. La celda unitaria se muestra en las proyecciones a lo largo de los ejes \(a\), \(b\) y \(c\) en la Figura \ref{fig:xenotimo_uc}.

  \begin{figure}[H]
   \centering
    \begin{subfigure}[b]{0.3\linewidth}
    \includegraphics[width=\linewidth]{Imagenes/2_XENOTIMO/ATOMS/XENOTIMO_a.png}
    \caption{Eje \(a\)}
    \label{fig:xenotimo_atoms_a}
    \end{subfigure}
      \hfill
    \begin{subfigure}[b]{0.3\linewidth}
    \includegraphics[width=\linewidth]{Imagenes/2_XENOTIMO/ATOMS/XENOTIMO_b.png}
    \caption{Eje \(b\)}
    \label{fig:xenotimo_atoms_b}
    \end{subfigure}
      \hfill
    \begin{subfigure}[b]{0.3\linewidth}
    \includegraphics[width=\linewidth]{Imagenes/2_XENOTIMO/ATOMS/XENOTIMO_c.png}
    \caption{Eje \(c\)}
    \label{fig:xenotimo_atoms_c}
    \end{subfigure}
 \caption{Proyecciones de la celdilla unidad}
 \label{fig:xenotimo_uc}
 \end{figure}

 \vspace{0.5cm}
 
 \begin{minipage}[c]{0.70\textwidth}
    \setlength{\parskip}{1em} % Mantiene espacio entre párrafos
    En cuanto al entorno de coordinación del fósforo, este forma un tetraedro \ce{PO4} con cuatro longitudes de enlace idénticas de \SI{1.533}{\angstrom}, lo que resulta en un índice de distorsión de la longitud de enlace de 0.0000. Aunque mantiene la simetría local del sitio \(4b\) (\(\bar{4}m2\)) al igual que el silicio en el Zircón, el tetraedro de fosfato es notablemente menos distorsionado angularmente: presenta una varianza del ángulo de enlace de \SI{21.48}{\degree\squared} (frente a los \SI{97.71}{\degree\squared} del silicato) y un volumen más reducido de \SI{1.84}{\cubic\angstrom}, consistente con el menor radio iónico del \ce{P^{5+}}.
 \end{minipage}% <--- ¡IMPORTANTE: No borres este símbolo de porcentaje!
   \hfill
  \begin{minipage}[c]{0.25\textwidth} % Columna de IMAGEN (Derecha)
   \centering
    \begin{figure}[H]
        \includegraphics[width=\linewidth]{Imagenes/2_XENOTIMO/XENOTIMO_Td.png}
        \captionsetup{justification=centering}
       \caption{Tetraedro \ce{PO4} del Xenótimo}
    \label{fig:xenotimo_td_fixed}
    \end{figure}
  
 \end{minipage}

  \vspace{0.5cm}

 \begin{minipage}[C]{0.3\textwidth} % Columna de TEXTO (Izquierda)
   \setlength{\parskip}{1em} % Mantiene espacio entre párrafos
    \centering
    \begin{figure}[H]
     \includegraphics[width=\linewidth]{Imagenes/2_XENOTIMO/XENOTIMO_Dode.png}
     \caption{Dodecaedro \ce{YO8} del Xenótimo}
     \label{fig:xenotimo_dode_fixed}
    \end{figure}
  
 \end{minipage}
    \hfill
 \begin{minipage}[c]{0.65\textwidth}
    \setlength{\parskip}{1em} % Mantiene espacio entre párrafos
    Por otro lado, el catión \ce{Y^{3+}} se encuentra en una coordinación 8 formando un dodecaedro \ce{YO8}. Este poliedro exhibe dos conjuntos de distancias de enlace, con cuatro enlaces más cortos de \SI{2.311}{\angstrom} y cuatro más largos de \SI{2.391}{\angstrom}, resultando en una distancia media de \SI{2.351}{\angstrom}. El índice de distorsión de la longitud de enlace es de 0.0169, un valor inferior al observado para el \ce{Zr^{4+}} en la estructura del Zircón (0.0318), lo que indica un entorno ligeramente más regular para el itrio. Debido al mayor tamaño del catión trivalente, el volumen del poliedro es significativamente mayor, alcanzando los \SI{23.30}{\cubic\angstrom} (ver Figuras \ref{fig:xenotimo_big_a} y \ref{fig:xenotimo_big_c}).
  \end{minipage}

  \vspace{0.5cm}

 En esta estructura, los iones \ce{Y^3+} están coordinados por ocho átomos de oxígeno, formando poliedros \ce{YO8} que se disponen alrededor de los tetraedros \ce{PO4}. Los tetraedros \ce{PO4} están orientados de manera que sus vértices apuntan hacia los centros de los poliedros \ce{YO8}, creando una red tridimensional interconectada. Esta disposición resulta en una estructura robusta y estable, característica del Xenótimo. Las representaciones de los poliedros \ce{YO8} y \ce{PO4}, así como del esqueleto de la estructura, se muestran en las Figuras \ref{fig:xenotimo_big_a}, \ref{fig:xenotimo_big_b} y \ref{fig:xenotimo_big_c}.

 \begin{figure}[H]
   \centering
    \begin{subfigure}[b]{0.45\linewidth}
    \includegraphics[width=\linewidth]{Imagenes/2_XENOTIMO/POLYHEDRA/XENOTIMO_a.png}
    \caption{Proyección de los poliedros}
    \end{subfigure}
      \hfill
    \begin{subfigure}[b]{0.45\linewidth}
    \includegraphics[width=\linewidth]{Imagenes/2_XENOTIMO/WIREFRAME/XENOTIMO_a.png}
    \caption{Proyección del esqueleto}
    \end{subfigure}
 \caption{Representaciones de los poliedros y del esqueleto en la celdilla unitaria del Xenótimo a lo largo del eje \(a\).}
 \label{fig:xenotimo_big_a}
 \end{figure}

 \begin{figure}[H]
   \centering
    \begin{subfigure}[b]{0.45\linewidth}
    \includegraphics[width=\linewidth]{Imagenes/2_XENOTIMO/POLYHEDRA/XENOTIMO_b.png}
    \caption{Proyección a lo largo del eje \(b\).}
    \end{subfigure}
      \hfill
    \begin{subfigure}[b]{0.45\linewidth}
    \includegraphics[width=\linewidth]{Imagenes/2_XENOTIMO/WIREFRAME/XENOTIMO_b.png}
    \caption{Proyección a lo largo del eje \(b\).}
    \end{subfigure}
 \caption{Representaciones apliadas de los poliedros y esqueletos del Xenótimo a lo largo del eje \(b\).}
 \label{fig:xenotimo_big_b}
 \end{figure}
 

 \begin{figure}[H]
 \centering
    \begin{subfigure}[b]{0.45\linewidth}
    \includegraphics[width=\linewidth]{Imagenes/2_XENOTIMO/POLYHEDRA/XENOTIMO_c.png}
    \caption{Proyección de los poliedros}
    \end{subfigure}
      \hfill
    \begin{subfigure}[b]{0.45\linewidth}
    \includegraphics[width=\linewidth]{Imagenes/2_XENOTIMO/WIREFRAME/XENOTIMO_c.png}
    \caption{Proyección del esqueleto}
    \end{subfigure}
 \caption{Representaciones de los poliedros y del esqueleto en la celdilla unitaria del Xenótimo a lo largo del eje \(c\).}
 \label{fig:xenotimo_big_c}
 \end{figure}
 

  \begin{figure}[H]
    \centering
    \includegraphics[width=0.8\textwidth]{Imagenes/2_XENOTIMO/XENOTIMO.png}
      \caption{Representación de los poliedros en perspectiva del Xenótimo.} 
    \label{fig:xenotimo_pol_perspective}
 \end{figure}

 \section{Monacita (\texorpdfstring{\ce{LnPO4}}{LnPO4}, Ln = La-Gd)}
   Los parámetros cristalográficos para la representación de la Monacita \ce{CePO4} se obtienen de ``Crystallography Open Database'' \footnote{\url{https://www.crystallography.net/cod/9016405.html}} \cite{mooney1948crystal}.

   La estructura de la Monacita se caracteriza por un sistema cristalino monoclínico con grupo espacial \(P2_1/n\) (No. 14). Lo que indica una celda de tipo primitiva (\(P\)), \(2_1\) indica un eje helicoidal de \(180^\circ\) con una traslación de 1/2 de la altura de la celda unitaria y \(n\) indica un plano de simetría diagonal. El grupo puntual \(2/m\), \(2\) indica un eje de rotación de \(180^\circ\) y \(m\) indica un plano de simetría perpendicular al eje principal \(b\).

   Las coordenadas atómicas, factores de ocupación y parámetros térmicos isotrópicos para la estructura de la Monacita se presentan en la Tabla \ref{tab:coord_monacita}.

   \begin{table}[H]
     \centering
     \caption{Coordenadas atómicas, factores de ocupación y parámetros térmicos isotrópicos para la estructura de la Monacita.}
     \label{tab:coord_monacita}
     \vspace{0.1cm}
     \begin{tabular}{l c c c c c c c}
     \toprule
     Átomo & \(x\) & \(y\) & \(z\) & Ocup & \(U\) & Pos. Wyckoff & Sim. \\
     \midrule
     Ce & 0.292 & 0.156 & 0.083 & 1 & 0.000 & \(4e\) & \(1\) \\
     P  & 0.292 & 0.156 & 0.581 & 1 & 0.000 & \(4e\) & \(1\) \\
     O1 & 0.211 & 0.990 & 0.423 & 1 & 0.000 & \(4e\) & \(1\) \\
     O2 & 0.374 & 0.323 & 0.464 & 1 & 0.000 & \(4e\) & \(1\) \\
     O3 & 0.467 & 0.086 & 0.765 & 1 & 0.000 & \(4e\) & \(1\) \\
     O4 & 0.116 & 0.235 & 0.680 & 1 & 0.000 & \(4e\) & \(1\) \\
     \bottomrule
     \end{tabular}
    \end{table}

    Los parámetros de la celda unitaria son \(a =6.76,  b= 7.00, c= 6.44\) \si{\angstrom} y los ángulos \(\alpha = \gamma = 90^\circ\), \(\beta \neq 90^\circ\). El volumen de la celda unitaria es \(296.158629\) \si{\cubic\angstrom}. Cada celda unitaria contiene 4 unidades de fórmula (\(Z=4\)), por lo tanto la fórmula empírica es \ce{Ce4P4O16}. La celda unitaria se muestra en las proyecciones a lo largo de los ejes \(a\), \(b\) y \(c\) en la Figura \ref{fig:monacita_uc}.

   \begin{figure}[H]
   \centering
    \begin{subfigure}[b]{0.3\linewidth}
    \includegraphics[width=\linewidth]{Imagenes/3_MONACITA/ATOMS/MONACITA_a.png}
    \caption{Eje \(a\)}
    \label{fig:monacita_atoms_a}
    \end{subfigure}
      \hfill
    \begin{subfigure}[b]{0.3\linewidth}
    \includegraphics[width=\linewidth]{Imagenes/3_MONACITA/ATOMS/MONACITA_b.png}
    \caption{Eje \(b\)}
    \label{fig:monacita_atoms_b}
    \end{subfigure}
      \hfill
    \begin{subfigure}[b]{0.3\linewidth}
    \includegraphics[width=\linewidth]{Imagenes/3_MONACITA/ATOMS/MONACITA_c.png}
    \caption{Eje \(c\)}
    \label{fig:monacita_atoms_c}
    \end{subfigure}
 \caption{Proyecciones de la celdilla unidad}
 \label{fig:monacita_uc}
 \end{figure}

 \vspace{0.5cm}

\begin{minipage}[c]{0.70\textwidth} % Columna de TEXTO (Izquierda)
    \setlength{\parskip}{1em}
    El entorno de coordinación del fósforo conforma un tetraedro \ce{PO4} extremadamente regular y rígido. Las distancias de enlace P--O son notablemente homogéneas, con un valor promedio de \SI{1.56}{\angstrom} y una varianza mínima. Esta uniformidad se refleja en un índice de distorsión de longitud de enlace muy bajo (0.0061).

    El análisis geométrico confirma la idealidad del poliedro, presentando una elongación cuadrática de 1.0004 y un número de coordinación efectivo de 3.99, prácticamente idéntico al valor teórico de 4. El volumen calculado para este tetraedro es de \SI{1.95}{\cubic\angstrom}, actuando como la unidad estructural estable de la red.
\end{minipage}%
    \hfill
\begin{minipage}[c]{0.25\textwidth} % Columna de IMAGEN (Derecha)
    \centering
    \begin{figure}[H]
        % REEMPLAZA CON EL NOMBRE DE TU ARCHIVO REAL
        \includegraphics[width=\linewidth]{Imagenes/3_MONACITA/MONACITA_Td.png}
        \captionsetup{justification=centering}
        \caption{Tetraedro \ce{PO4}}
        \label{fig:monacita_td} % Puedes cambiar la etiqueta si quieres
    \end{figure}
\end{minipage}

\vspace{0.5cm}

\begin{minipage}[c]{0.25\textwidth} % Columna de IMAGEN (Izquierda)
    \centering
    \begin{figure}[H]
        % REEMPLAZA CON EL NOMBRE DE TU ARCHIVO REAL
        \includegraphics[width=\linewidth]{Imagenes/3_MONACITA/MONACITA_ene.png}
        \captionsetup{justification=centering}
        \caption{Eneacedro \ce{CeO9}}
        \label{fig:cerio_poly}
    \end{figure}
\end{minipage}%
    \hfill
\begin{minipage}[c]{0.70\textwidth} % Columna de TEXTO (Derecha)
    \setlength{\parskip}{1em}
    El entorno de coordinación del cerio conforma un poliedro de nueve vértices (eneacedro) significativamente distorsionado, con un volumen calculado de \SI{31.85}{\cubic\angstrom}. A diferencia de la rigidez del fosfato, este sitio presenta un índice de distorsión de longitud de enlace de 0.05599.

    Esta irregularidad se manifiesta claramente en la dispersión de las distancias de enlace Ce--O, que varían desde \SI{2.26}{\angstrom} para el enlace más corto hasta \SI{2.75}{\angstrom} para el más largo, con un promedio de \SI{2.55}{\angstrom}. Dicha variabilidad resulta en un número de coordinación efectivo de 6.76, lo que indica que la esfera de coordinación real está más comprimida que la geométrica.
\end{minipage}

\vspace{0.5cm}
 
 En esta estructura, los iones \ce{Ln^3+} están coordinados por nueve átomos de oxígeno, formando poliedros \ce{LnO9} que se disponen alrededor de los tetraedros \ce{PO4}. 
 
 Las representaciones de los poliedros \ce{LnO9} y \ce{PO4}, así como del esqueleto de la estructura, se muestran en las Figuras \ref{fig:monacita_big_a}, \ref{fig:monacita_big_b} y \ref{fig:monacita_big_c}.

 \begin{figure}[H]
 \centering
    \begin{subfigure}[b]{0.45\linewidth}
    \includegraphics[width=\linewidth]{Imagenes/3_MONACITA/POLYHEDRA/MONACITA_a.png}
    \caption{Proyección de los poliedros}
    \end{subfigure}
      \hfill
    \begin{subfigure}[b]{0.45\linewidth}
    \includegraphics[width=\linewidth]{Imagenes/3_MONACITA/WIREFRAME/MONACITA_a.png}
    \caption{Proyección del esqueleto}
    \end{subfigure}
 \caption{Representaciones de los poliedros y del esqueleto en la celdilla unitaria del Monacita a lo largo del eje \(a\).}
 \label{fig:monacita_big_a}
 \end{figure}

 \begin{figure}[H]
 \centering
    \begin{subfigure}[b]{0.45\linewidth}
    \includegraphics[width=\linewidth]{Imagenes/3_MONACITA/POLYHEDRA/MONACITA_b.png}
    \caption{Proyección a lo largo del eje \(b\).}
    \end{subfigure}
      \hfill
    \begin{subfigure}[b]{0.45\linewidth}
    \includegraphics[width=\linewidth]{Imagenes/3_MONACITA/WIREFRAME/MONACITA_b.png}
    \caption{Proyección a lo largo del eje \(b\).}
    \end{subfigure}
 \caption{Representaciones apliadas de los poliedros y esqueletos del Monacita a lo largo del eje \(b\).}
 \label{fig:monacita_big_b}
 \end{figure}
 

 \begin{figure}[H]
 \centering
    \begin{subfigure}[b]{0.45\linewidth}
    \includegraphics[width=\linewidth]{Imagenes/3_MONACITA/POLYHEDRA/MONACITA_c.png}
    \caption{Proyección de los poliedros}
    \end{subfigure}
      \hfill
    \begin{subfigure}[b]{0.45\linewidth}
    \includegraphics[width=\linewidth]{Imagenes/3_MONACITA/WIREFRAME/MONACITA_c.png}
    \caption{Proyección del esqueleto}
    \end{subfigure}
 \caption{Representaciones de los poliedros y del esqueleto en la celdilla unitaria del Monacita a lo largo del eje \(c\).}
 \label{fig:monacita_big_c}
 \end{figure}
 

  \begin{figure}[H]
    \centering
    \includegraphics[width=0.8\textwidth]{Imagenes/3_MONACITA/MONACITA.png}
    \caption{Representación de los poliedros en perspectiva del Monacita.} 
    \label{fig:monacita_pol_perspective}
 \end{figure}
 
 \section{Churchita (\texorpdfstring{\ce{LnPO4*nH2O}}{LnPO4*nH2O}, Ln= Y, Tb-Lu)}
   Los parámetros cristalográficos para la representación de la Churchita \ce{YPO4*2H2O} se obtienen de ``Crystallography Open Database'' \footnote{\url{https://www.mindat.org/min-1047.html}} \cite{kohlmann1994structure} .

   La estructura de la Churchita se caracteriza por un sistema cristalino monoclínico con grupo espacial \textit{C2/c} (No. 15). Lo que indica una celda de tipo centrada en el cuerpo (\(C\)), \(2\) indica un eje de rotación de \(180^\circ\) y \(c\) indica un plano de simetría diagonal. El grupo puntual \(2/m\), \(2\) indica un eje de rotación de \(180^\circ\) y \(m\) indica un plano de simetría perpendicular al eje principal \(b\).

   Las coordenadas atómicas, factores de ocupación y parámetros térmicos isotrópicos para la estructura de la Churchita se presentan en la Tabla \ref{tab:coord_churchita}.
  
   \begin{table}[H]
     \centering
     \caption{Coordenadas atómicas, factores de ocupación y parámetros térmicos isotrópicos para la estructura de la Churchita.}
     \label{tab:coord_churchita}
     \vspace{0.1cm}
     \begin{tabular}{l c c c c c c c}
     \toprule
     Átomo & \(x\) & \(y\) & \(z\) & Ocup & \(U\) & Pos. Wyckoff & Sim. \\
     \midrule
     Y    & 0.2500 & 0.8289 & 0.0000 & 1 & 0.007 & \(4e\) & \(2\) \\
     P    & 0.2500 & 0.3307 & 0.0000 & 1 & 0.014 & \(4e\) & \(2\) \\
     O1   & 0.3020 & 0.3857 & 0.2240 & 1 & 0.022 & \(8f\) & \(1\) \\
     O2   & 0.5060 & 0.2714 & 0.0840 & 1 & 0.018 & \(8f\) & \(1\) \\
     Ow   & 0.6300 & 0.0680 & 0.2180 & 1 & 0.021 & \(8f\) & \(1\) \\
     \bottomrule
    \end{tabular}
 \end{table}
    
 Los parámetros de la celda unitaria son \(a \neq b \neq c\), con \(a= 5.5780, b= 15.0060, c= 6.2750\) \si{\angstrom} y los ángulos \(\alpha = \gamma = 90^\circ\), \(\beta = 117.83^\circ\). El volumen de la celda unitaria es \(464.488323\) \si{\cubic\angstrom}. Cada celda unitaria contiene 4 unidades de fórmula (\(Z=4\)), por lo tanto la fórmula empírica es \ce{Y4P4O16*8H2O}. La celda unitaria se muestra en las proyecciones a lo largo de los ejes \(a\), \(b\) y \(c\) en la Figura \ref{fig:churchita_uc}.

 \begin{figure}[H]
    \centering
    % --- COLUMNA IZQUIERDA: Imágenes A y B (una encima de otra) ---
    \begin{minipage}[c]{0.5\linewidth} % Ocupa el 45% del ancho de la página
        \centering
        % Subfigura A
        \begin{subfigure}[b]{\linewidth}
            \centering
            \includegraphics[width=\linewidth]{Imagenes/4_CHURCHITA/ATOMS/CHURCHITA_a.png}
            \caption{Eje \(a\)}
            \label{fig:churchita_atoms_a}
        \end{subfigure}
        
        \hfill % <--- Este es el espacio vertical entre A y B
        
        % Subfigura B
        \begin{subfigure}[b]{0.8\linewidth}
            \centering
            \includegraphics[width=\linewidth]{Imagenes/4_CHURCHITA/ATOMS/CHURCHITA_b.png}
            \caption{Eje \(b\)}
            \label{fig:churchita_atoms_b}
        \end{subfigure}
    \end{minipage}
    \hfill % <--- Esto empuja la columna derecha hacia el borde opuesto
    % --- COLUMNA DERECHA: Imagen C (sola) ---
    \begin{minipage}[c]{0.25\linewidth} % Ocupa el 45% del ancho
        \centering
        % Subfigura C
        \begin{subfigure}[b]{\linewidth}
            \centering
            \includegraphics[width=\linewidth]{Imagenes/4_CHURCHITA/ATOMS/CHURCHITA_c.png}
            \caption{Eje \(c\)}
            \label{fig:churchita_atoms_c}
        \end{subfigure}
    \end{minipage}
    
    \caption{Proyecciones de la celdilla unidad}
    \label{fig:churchita_uc}
 \end{figure}

 \vspace{0.5cm}

\begin{minipage}[c]{0.70\textwidth} % Columna de TEXTO
   \setlength{\parskip}{1em}
   El entorno de coordinación del fósforo conforma un tetraedro \ce{PO4} que, a diferencia de la alta simetría observada en el Xenótimo, presenta una ligera distorsión. Se distinguen dos conjuntos de distancias de enlace P--O: dos enlaces P--O1 de \SI{1.536}{\angstrom} y dos enlaces P--O2 ligeramente más largos de \SI{1.551}{\angstrom}. Esta leve irregularidad se traduce en un índice de distorsión de la longitud de enlace de 0.0048 y una varianza angular de \SI{18.96}{\degree\squared}. El volumen del tetraedro es de \SI{1.88}{\cubic\angstrom}, manteniendo la rigidez característica del grupo fosfato con una distancia media de enlace de \SI{1.544}{\angstrom}.
\end{minipage}
\hfill
\begin{minipage}[c]{0.25\textwidth} % Columna de IMAGEN (Opcional)
   \centering

   \begin{figure}[H]
     \includegraphics[width=\linewidth]{Imagenes/4_CHURCHITA/CHURCHITA_Td.png}
     \captionsetup{justification=centering}
     \caption{Tetraedro \ce{PO4} de la Churchita}
     \label{fig:churchita_td}
   \end{figure}

\end{minipage}

\vspace{0.5cm}

\begin{minipage}[c]{0.3\textwidth} % Columna de IMAGEN (Izquierda)
   \centering
   \begin{figure}[H]
     \includegraphics[width=\linewidth]{Imagenes/4_CHURCHITA/CHURCHITA_Dode.png}
     \captionsetup{justification=centering}
     \caption{Poliedro \ce{YPO4* nH2O} de la Churchita}
   \label{fig:churchita_dode}
   \end{figure}
\end{minipage}%
   \hfill 
\begin{minipage}[c]{0.65\textwidth} % Texto a todo lo ancho
   \setlength{\parskip}{1em}
   Por su parte, el catión itrio (\ce{Y^{3+}}) presenta un número de coordinación 8, conformando un poliedro de geometría compleja que integra tanto átomos de oxígeno de los grupos fosfato como moléculas de agua estructural. El entorno de coordinación es notablemente heterogéneo, exhibiendo cuatro pares de distancias de enlace diferenciadas: los enlaces más cortos corresponden a Y--O2 (\SI{2.251}{\angstrom}), seguidos por la interacción con las moléculas de agua Y--Wat3 (\SI{2.360}{\angstrom}), y finalmente los enlaces más largos Y--O1 (\SI{2.435}{\angstrom}) y el segundo conjunto de Y--O2 (\SI{2.469}{\angstrom}). Esta variabilidad en las longitudes de enlace genera un índice de distorsión de 0.0308, superior al observado en la fase anhidra (Xenótimo), y una distancia media de enlace de \SI{2.379}{\angstrom}. El volumen de este poliedro hidratado es de \SI{23.81}{\cubic\angstrom}. La presencia de las moléculas de agua en la esfera de coordinación es el factor clave que diferencia estructuralmente a la Churchita del Xenótimo, permitiendo la formación de una red de enlaces de hidrógeno que estabiliza la estructura monoclínica.
\end{minipage}

\vspace{0.5cm}

 En esta estructura, los iones \ce{Y^3+} están coordinados por ocho átomos de oxígeno y moléculas de agua, formando poliedros \ce{YPO4* nH2O} que se disponen alrededor de los tetraedros \ce{PO4}.
 
 Las representaciones de los poliedros \ce{YPO4* nH2O} y \ce{PO4}, así como del esqueleto de la estructura, se muestran en las Figuras \ref{fig:churchita_big_a}, \ref{fig:churchita_big_b} y \ref{fig:churchita_big_c}.


 \begin{figure}[H]
 \centering
    \begin{subfigure}[b]{0.45\linewidth}
    \includegraphics[width=\linewidth]{Imagenes/4_CHURCHITA/POLYHEDRA/CHURCHITA_a.png}
    \caption{Proyección de los poliedros}
    \end{subfigure}
      \hfill
    \begin{subfigure}[b]{0.45\linewidth}
    \includegraphics[width=\linewidth]{Imagenes/4_CHURCHITA/WIREFRAME/CHURCHITA_a.png}
    \caption{Proyección del esqueleto}
    \end{subfigure}
 \caption{Representaciones de los poliedros y del esqueleto en la celdilla unitaria del Churchita a lo largo del eje \(a\).}
 \label{fig:churchita_big_a}
 \end{figure}

 \begin{figure}[H]
 \centering
    \begin{subfigure}[b]{0.45\linewidth}
    \includegraphics[width=\linewidth]{Imagenes/4_CHURCHITA/POLYHEDRA/CHURCHITA_b.png}
    \caption{Proyección a lo largo del eje \(b\).}
    \end{subfigure}
      \hfill
    \begin{subfigure}[b]{0.45\linewidth}
    \includegraphics[width=\linewidth]{Imagenes/4_CHURCHITA/WIREFRAME/CHURCHITA_b.png}
    \caption{Proyección a lo largo del eje \(b\).}
    \end{subfigure}
 \caption{Representaciones apliadas de los poliedros y esqueletos del Churchita a lo largo del eje \(b\).}
 \label{fig:churchita_big_b}
 \end{figure}
 

 \begin{figure}[H]
 \centering
    \begin{subfigure}[b]{0.45\linewidth}
    \includegraphics[width=\linewidth]{Imagenes/4_CHURCHITA/POLYHEDRA/CHURCHITA_c.png}
    \caption{Proyección de los poliedros}
    \end{subfigure}
      \hfill
    \begin{subfigure}[b]{0.45\linewidth}
    \includegraphics[width=\linewidth]{Imagenes/4_CHURCHITA/WIREFRAME/CHURCHITA_c.png}
    \caption{Proyección del esqueleto}
    \end{subfigure}
 \caption{Representaciones de los poliedros y del esqueleto en la celdilla unitaria del Churchita a lo largo del eje \(c\).}
 \label{fig:churchita_big_c}
 \end{figure}
 

  \begin{figure}[H]
    \centering
    \includegraphics[width=0.8\textwidth]{Imagenes/4_CHURCHITA/CHURCHITA.png}
    \caption{Representación de los poliedros en perspectiva del Churchita.} 
    \label{fig:churchita_pol_perspective}
 \end{figure}

 \section{Rhabdofano (\texorpdfstring{\ce{LnPO4*nH2O}}{LnPO4*nH2O}, Ln= Y, Tb-Lu)}
   Los parámetros cristalográficos para la representación del Rhabdofano con \ce{CePO4*0.67H2O} se obtienen de ``Mindat.org'' \footnote{\url{https://www.mindat.org/min-3397.html}} \cite{Mooney_rhabdophane}.

   La estructura del Rhabdofano se caracteriza por un sistema cristalino hexagonal con grupo espacial \({P 6_2 2 2}\) (No. 180). Lo que indica una celda de tipo primitiva (\(P\)), \(6_2\) indica un eje helicoidal de \(60^\circ\) con una traslación de 2/6 de la altura de la celda unitaria, y los dos \(2\) indican ejes de rotación de \(180^\circ\) perpendiculares al eje principal \(c\). El grupo puntual \(6mm\), \(6\) indica un eje de rotación de \(60^\circ\) y las dos \(m\) indican planos de simetría verticales al eje principal \(c\), uno pasando por los ejes \(a\) y \(b\) y otro pasando por las diagonales del plano \(ab\).

   Las coordenadas atómicas, factores de ocupación y parámetros térmicos isotrópicos para la estructura del Rhabdofano se presentan en la Tabla \ref{tab:coord_rhabdofano}.

   \begin{table}[H]
      \centering
      \caption{Parámetros estructurales}
      \label{tab:coord_rhabdofano}
     \begin{tabular}{ l c c c c c c c}
        \toprule
        Átomo & \(x\) & \(y\) & \(z\) & Ocup. & \(U\) & Pos. Wyckoff & Sim. \\
        \midrule
        Ce & 0.500 & 0.000 & 0.000 & 1 & 0.000 & \(3c\)  & \(222\) \\
        P  & 0.500 & 0.000 & 0.500 & 1 & 0.000 & \(3d\)  & \(222\) \\
        O  & 0.446 & 0.147 & 0.360 & 1 & 0.000 & \(12k\) & \(1\) \\
        Ow & 0.000 & 0.000 & 0.000 & 1 & 0.000 & \(3a\)  & \(222\) \\
        \bottomrule
     \end{tabular}
   \end{table}

    Los parámetros de la celda unitaria son \(a = b \neq c\) con \(a=7.0550 \si{\angstrom}\) y los ángulos \(\alpha = \beta = 90^\circ\), \(\gamma = 120^\circ\). El volumen de la celda unitaria es \(277.551182\) \si{\cubic\angstrom}. Cada celda unitaria contiene 3 unidades de fórmula (\(Z=3\)), por lo tanto la fórmula empírica es \ce{Ce3P3O12*2H2O}. La celda unitaria se muestra en las proyecciones a lo largo de los ejes \(a\), \(b\) y \(c\) en la Figura \ref{fig:rhabdofano_uc}.

   \begin{figure}[H]
    \centering
    \begin{subfigure}[b]{0.3\linewidth}
    \includegraphics[width=\linewidth]{Imagenes/5_RHABDOFANO/ATOMS/RHABDOFANO_a.png}
    \caption{Eje \(a\)}
    \label{fig:rhabdofano_atoms_a}
    \end{subfigure}
      \hfill
    \begin{subfigure}[b]{0.3\linewidth}
    \includegraphics[width=\linewidth]{Imagenes/5_RHABDOFANO/ATOMS/RHABDOFANO_b.png}
    \caption{Eje \(b\)}
    \label{fig:rhabdofano_atoms_b}
    \end{subfigure}
      \hfill
    \begin{subfigure}[b]{0.3\linewidth}
    \includegraphics[width=\linewidth]{Imagenes/5_RHABDOFANO/ATOMS/RHABDOFANO_c.png}
    \caption{Eje \(c\)}
    \label{fig:rhabdofano_atoms_c}
    \end{subfigure}
 \caption{Proyecciones de la celdilla unidad}
 \label{fig:rhabdofano_uc}
 \end{figure}

 \vspace{0.5cm}
 
\begin{minipage}[c]{0.70\textwidth} % Columna de TEXTO
   \setlength{\parskip}{1em}
   El entorno de coordinación del fósforo en el Rhabdofano destaca por su extraordinaria regularidad geométrica. A diferencia de la fase hidratada monoclínica (Churchita), este tetraedro \ce{PO4} es cristalográficamente perfecto en cuanto a sus longitudes de enlace, presentando cuatro distancias P--O idénticas de \SI{1.558}{\angstrom}. Esta simetría ideal se refleja en un índice de distorsión de longitud de enlace nulo (0.0000) y una elongación cuadrática de 1.0000. La varianza angular es prácticamente despreciable (\SI{0.018}{\degree\squared}), lo que confirma que el poliedro no sufre deformaciones angulares significativas. El volumen calculado para este tetraedro es de \SI{1.94}{\cubic\angstrom}, ligeramente mayor que el observado en las fases anhidras como el Xenótimo (\SI{1.84}{\cubic\angstrom}).
\end{minipage}
    \hfill
\begin{minipage}[c]{0.25\textwidth}
  \begin{figure}[H]
    \centering
    \includegraphics[width=\linewidth]{Imagenes/5_RHABDOFANO/RHABDOFANO_Td.png}
    \captionsetup{justification=centering}
    \caption{Tetraedro \ce{PO4} del Rhabdofano}
    \label{fig:rhabdofano_td}
  \end{figure}
\end{minipage}%

\vspace{0.5cm}

\begin{minipage}[c]{0.3\textwidth} % Columna de IMAGEN (Izquierda)
 \centering
   \begin{figure}[H]
    \includegraphics[width=\linewidth]{Imagenes/5_RHABDOFANO/RHABDOFANO_Dode.png} 
   \captionsetup{justification=centering}
   \caption{Poliedro \ce{CeO8} del Rhabdofano}
   \label{fig:rhabdofano_Dode}
   \end{figure}
\end{minipage}%
   \hfill
\begin{minipage}[c]{0.65\textwidth} % Texto a todo lo ancho
   \setlength{\parskip}{1em}
   Por otro lado, el catión cerio (\ce{Ce^{3+}}) se encuentra coordinado por ocho átomos de oxígeno pertenecientes a los grupos fosfato, conformando un poliedro \ce{CeO8} con geometría de antiprisma cuadrado distorsionado. A diferencia del entorno del fósforo, este sitio catiónico presenta una fuerte distorsión, evidenciada por un índice de 0.0631. El análisis de las distancias de enlace revela dos conjuntos claramente diferenciados: cuatro enlaces Ce--O más cortos de \SI{2.330}{\angstrom} y cuatro enlaces significativamente más largos de \SI{2.644}{\angstrom}. Esta disparidad resulta en una distancia media de enlace de \SI{2.487}{\angstrom} y un número de coordinación efectivo de 6.64, lo que sugiere que, aunque geométricamente hay 8 oxígenos vecinos, la interacción es mucho más fuerte con los cuatro más cercanos. El volumen del poliedro es de \SI{25.95}{\cubic\angstrom}, siendo el más voluminoso de las estructuras estudiadas, consistente con la naturaleza abierta de la estructura del Rhabdofano que aloja canales zeolíticos.
\end{minipage}

\vspace{0.5cm}

 En esta estructura, los iones \ce{Ce^{3+}} están coordinados por ocho átomos de oxígeno y moléculas de agua, formando poliedros \ce{CeO8* 0.67H2O} que se disponen alrededor de los tetraedros \ce{PO4}. 

 Las representaciones de los poliedros \ce{CeO8* 0.67H2O} y \ce{PO4}, así como del esqueleto de la estructura, se muestran en las Figuras \ref{fig:rhabdofano_big_a}, \ref{fig:rhabdofano_big_b} y \ref{fig:rhabdofano_big_c}.

 \begin{figure}[H]
 \centering
    \begin{subfigure}[b]{0.45\linewidth}
    \includegraphics[width=\linewidth]{Imagenes/5_RHABDOFANO/POLYHEDRA/RHABDOFANO_a.png}
    \caption{Proyección de los poliedros}
    \end{subfigure}
      \hfill
    \begin{subfigure}[b]{0.45\linewidth}
    \includegraphics[width=\linewidth]{Imagenes/5_RHABDOFANO/WIREFRAME/RHABDOFANO_a.png}
    \caption{Proyección del esqueleto}
    \end{subfigure}
 \caption{Representaciones de los poliedros y del esqueleto en la celdilla unitaria del Rhabdofano a lo largo del eje \(a\).}
 \label{fig:rhabdofano_big_a}
 \end{figure}

 \begin{figure}[H]
 \centering
    \begin{subfigure}[b]{0.45\linewidth}
    \includegraphics[width=\linewidth]{Imagenes/5_RHABDOFANO/POLYHEDRA/RHABDOFANO_b.png}
    \caption{Proyección a lo largo del eje \(b\).}
    \end{subfigure}
      \hfill
    \begin{subfigure}[b]{0.45\linewidth}
    \includegraphics[width=\linewidth]{Imagenes/5_RHABDOFANO/WIREFRAME/RHABDOFANO_b.png}
    \caption{Proyección a lo largo del eje \(b\).}
    \end{subfigure}
 \caption{Representaciones apliadas de los poliedros y esqueletos del Rhabdofano a lo largo del eje \(b\).}
 \label{fig:rhabdofano_big_b}
 \end{figure}
 

 \begin{figure}[H]
 \centering
    \begin{subfigure}[b]{0.45\linewidth}
    \includegraphics[width=\linewidth]{Imagenes/5_RHABDOFANO/POLYHEDRA/RHABDOFANO_c.png}
    \caption{Proyección de los poliedros}
    \end{subfigure}
      \hfill
    \begin{subfigure}[b]{0.45\linewidth}
    \includegraphics[width=\linewidth]{Imagenes/5_RHABDOFANO/WIREFRAME/RHABDOFANO_c.png}
    \caption{Proyección del esqueleto}
    \end{subfigure}
 \caption{Representaciones de los poliedros y del esqueleto en la celdilla unitaria del Rhabdofano a lo largo del eje \(c\).}
 \label{fig:rhabdofano_big_c}
 \end{figure}
 

  \begin{figure}[H]
    \centering
    \includegraphics[width=0.8\textwidth]{Imagenes/5_RHABDOFANO/RHABDOFANO.png}
    \caption{Representación de los poliedros en perspectiva del Rhabdofano.} 
    \label{fig:rhabdofano_pol_perspective}
 \end{figure}



\chapter{Conclusiones}
 Resumen de hallazgos.


% D. REFERENCIAS BIBLIOGRÁFICAS (Num. Arábiga)

\clearpage
\addcontentsline{toc}{chapter}{Bibliografía} % Se añade al índice como un elemento no numerado
\printbibliography
\nocite{*}



\end{document}
